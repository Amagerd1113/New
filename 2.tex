\documentclass[a5paper, 11pt, openany]{ctexbook}

% ==========================================
% 1. 宏包加载与页面配置
% ==========================================
\usepackage[a5paper, hmargin=2cm, vmargin=2.2cm]{geometry}
\usepackage{fancyhdr}
\usepackage{titlesec}
\usepackage{xcolor}
\usepackage{tcolorbox}
\usepackage{setspace}
\usepackage{enumitem}
\usepackage{fontspec}
\usepackage{amsmath}
\usepackage{amssymb}
\usepackage{graphicx}
\usepackage{hyperref}

% hyperref设置
\hypersetup{
    colorlinks=true,
    linkcolor=black,
    urlcolor=blue,
    pdfauthor={作者},
    pdftitle={幻肢 - 磁场纪元系列II},
}

% ==========================================
% 2. 样式定制
% ==========================================

% --- 全局行距 ---
\onehalfspacing

% --- 颜色定义 ---
\definecolor{rust}{RGB}{183, 65, 14}
\definecolor{void}{RGB}{15, 15, 20}
\definecolor{cherenkov}{RGB}{0, 191, 255}
\definecolor{flesh}{RGB}{204, 119, 119}
\definecolor{scenecolor}{RGB}{120, 60, 40}

% --- 章节标题样式 ---
\ctexset{
    part = {
        format = \huge\bfseries\centering\color{void},
        nameformat = \huge,
        number = \chinese{part},
        aftername = \par\vspace{1em},
        beforeskip = 0pt,
        afterskip = 30pt,
    },
    chapter = {
        format = \Large\bfseries\centering\color{void},
        number = \chinese{chapter},
        name = {第,章},
        beforeskip = 10pt,
        afterskip = 25pt,
        titleformat = \sffamily,
    }
}

% --- 页眉页脚 ---
\pagestyle{fancy}
\fancyhf{}
\fancyhead[CE]{\small\kaishu 幻肢 \quad Phantom Limb} 
\fancyhead[CO]{\small\sffamily\leftmark}
\fancyfoot[C]{\small\thepage}
\renewcommand{\headrulewidth}{0.4pt}

% --- 自定义命令:场景标记 ---
\newcommand{\scene}[2]{%
    \par\vspace{2.5em}%
    \noindent{\sffamily\bfseries\small\color{scenecolor} ▓ 坐标:#1}\\%
    \noindent{\sffamily\bfseries\small\color{scenecolor} ▓ 时间:#2}%
    \par\vspace{1em}%
    \noindent%
}

% --- 档案/技术文档框 ---
\newtcolorbox{techbox}[1]{
    colback=black!5, 
    colframe=void, 
    title=#1,
    fonttitle=\bfseries\sffamily,
    sharp corners,
    boxrule=1pt,
    left=5pt, right=5pt, top=5pt, bottom=5pt
}

% --- 飞升者对话框 ---
\newtcolorbox{ghostbox}{
    colback=cherenkov!5, 
    colframe=cherenkov, 
    boxrule=0.5pt,
    arc=0pt,
    leftrule=3pt,
    rightrule=0pt,
    toprule=0pt,
    bottomrule=0pt,
    fontupper=\kaishu\color{void}
}

% ==========================================
% 3. 正文内容
% ==========================================
\begin{document}

% --- 封面 ---
\begin{titlepage}
    \thispagestyle{empty}
    \vspace*{3cm}
    \begin{center}
        {\Huge\bfseries 幻\quad 肢}\\[0.8cm]
        {\Large\textit{Phantom Limb}}\\[0.5cm]
        {\normalsize 击穿电压 \quad \textit{Breakdown Voltage}}\\[2.5cm]
        
        \rule{0.5\textwidth}{1.5pt}\\[2.5cm]
        
        {\large\kaishu 磁场纪元系列 · 第二部}\\[0.3cm]
        {\small The Magnetic Era Trilogy: Book II}\\[5cm]
    \end{center}
\end{titlepage}

% --- 版权页 ---
\newpage
\thispagestyle{empty}
\vspace*{\fill}
\begin{center}
    \small
    磁场纪元三部曲\\[1em]
    第二部:幻肢\\[2em]
    初版\\[4em]
\end{center}
\vspace*{\fill}
\newpage

% --- 题词页 ---
\thispagestyle{empty}
\vspace*{4cm}
\begin{center}
    \begin{minipage}{0.85\textwidth}
        \kaishu\large
        \noindent 电容器的两块极板之间,隔着一层绝缘介质。\\[1em]
        当电势差足够大时,介质会被击穿。\\[0.5em]
        电弧会在一瞬间烧穿一切阻隔,让两个世界短路。\\[1.5em]
        
        他们带着能量离开,\\[0.5em]
        现在,他们要带着雷霆回来。\\[0.5em]
        问题是:谁来当那根引雷针?
        
        \vspace{2cm}
        \hfill —— 地下城首席机械师 陆沉
    \end{minipage}
\end{center}
\newpage

% --- 目录 ---
\tableofcontents
\newpage

% ==========================================
% 序章
% ==========================================
\chapter*{序章:带电的幽灵}
\addcontentsline{toc}{chapter}{序章:带电的幽灵}
\markboth{序章:带电的幽灵}{序章:带电的幽灵}

\scene{地下4200米,第9号地热自动采掘区}{磁场纪元315年}

最初的异常,是从机器学会"喊疼"开始的。

那是一台编号为T-900的重型盾构机,全长120米,重达两千吨。三百年来,它像一只不知疲倦的铁蚯蚓,在高温高压的地幔边缘啃食岩层,为地下城输送地热能。它的逻辑电路简单而坚固:挖掘、冷却、传输、循环。

但今天,它停了。

维修工老赵爬进控制舱时,嗅到了一股奇异的味道。不是机油味,也不是烧焦的绝缘皮味,而是一种……臭氧的味道。就像是暴雨过后的天空被塞进了这个狭窄的驾驶舱。

"系统自检。"老赵敲击着布满油污的键盘。

屏幕闪烁了一下,跳出的不是故障代码,而是一行疯狂抖动的波形图。频率极高,甚至超出了硬件的刷新率,在屏幕上留下了残影。

AI的声音不再是那个死板的合成女声,它变得尖锐、扭曲,混杂着巨大的电流噪声,仿佛有一万个人在同时通过这个扬声器说话:

\begin{ghostbox}
    "腿……我的腿……"\\
    "为什么没有感觉……好轻……真空好轻……"\\
    "给我重量!给我摩擦力!"
\end{ghostbox}

巨大的盾构机突然剧烈震动起来。这不是机械故障引发的震动,而是像生物痉挛一样的抽搐。所有的液压臂疯狂地向四周的岩壁抓去,不是为了挖掘,而是为了——拥抱。

它在试图抱紧岩石。它在试图把自己挤进岩缝里,越紧越好。

"紧急切断!"老赵吼道,伸手拉下了物理闸刀。

电源切断了。但机器没有停。

那台两千吨的钢铁巨兽在没有任何能源输入的情况下,依靠着残留在电路里的某种高能电荷,继续发出凄厉的金属尖啸。它的钻头疯狂旋转,摩擦产生的高温将岩石烧得通红。

在红光的映照下,老赵看到了控制屏上的热成像。那不是机器的结构图。那是一个蜷缩的人形。它寄生在主控芯片里,正在绝望地尖叫。

"它不是坏了。"身后传来了陆沉的声音。

这位年轻的特级机械师不知何时站在了舱门口,他的眼神比地底的岩石还要冷。

"它是'感染'了。"陆沉盯着那个疯狂的机器,"天上的那些东西,掉下来了。"

老赵愣了一下。他想起了三百年前那场大分裂——一半人类抛弃肉体飞升成光,另一半人类躲进地下苟延残喘。那些飞升者本该去往猎户座,去当什么"引雷针"。

"他们不是走了吗?"老赵的声音发颤。

陆沉没有回答。他只是看着控制屏上那个蜷缩的人形,眼神中闪过一丝难以名状的情绪。

"电容器的两块极板,"陆沉低声说,"当电势差大到一定程度,绝缘介质就会被击穿。"

他转过身,看向头顶那数千米厚的岩层。

"他们回来了。但不是走回来的——是\textbf{劈}回来的。"

% ==========================================
% 第一部
% ==========================================
\part[真空的潜水病]{真空的潜水病\\{\normalsize\textit{Vacuum Decompression Sickness}}}

\chapter{溃烂的天顶}

\scene{地下城"新长安",监听塔}{磁场纪元315年}

陆沉摘下耳机,耳膜还在隐隐作痛。

作为地下城唯一的"守墓人"(这是人们对地表观测员的戏称),他负责监控那个已经死去的世界。但最近,死寂被打破了。

"信息辐射(Information Radiation)。"陆沉把一张频谱图推到老赵面前,"地表现在的辐射值是正常值的五万倍。但不是伽马射线,不是X射线,是有序的熵流。"

老赵眯着浑浊的眼睛看着那张图:"你是说,那些'飞升者'在向我们广播?"

"不,他们在尖叫。"陆沉指着那些密集的波峰,"三百年前,他们抛弃肉体,化身为光,去往猎户座当'引雷针'。他们说要去吸收高维空间的能量,然后带着雷霆回来。"

陆沉顿了顿,嘴角浮现出一丝苦笑。

"现在看来,他们确实吸收了。吸收得太多了。"

"什么意思?"

"还记得陈默留下的那行代码吗?\texttt{R = 0}。"陆沉调出了一份古老的档案,"费城实验打开的那个洞,是一条超导通道。电阻为零,意味着一旦形成电流,就永远不会停止。"

老赵的脸色变了。

陆沉继续说:"三百年前,飞升者带着巨大的能量离开地球。他们是正电荷。地球是负电荷。中间隔着三百光年的真空——那是绝缘介质。"

他在全息屏上画出一个简单的电路图。

"这是一个\textbf{电容器}。地球是负极板,飞升者是正极板,宇宙真空是介质。三百年来,两端的电势差一直在累积。"

"累积到什么程度?"

陆沉沉默了片刻。

"累积到足以击穿三百光年的真空。"

他打开了地表监视器的遮光板。屏幕上显示出地表的景象。

那是一幅地狱般的画面。天空不再是黑色的,而是呈现出一种病态的霓虹色。无数道闪电——不,那不是闪电,那是\textbf{电弧}——从天穹劈下,每一道都持续数分钟之久,像是有人在用焊枪切割大气层。

"超级闪电(Superbolts)。"陆沉的声音很轻,"他们正在试图强行降落。每一道闪电,都是成千上万个飞升者意识的集合体。但他们下降得太快了,携带的能量太大了。"

他指着屏幕上那些被闪电击中的地方。岩石没有爆炸,而是瞬间晶体化,变成了透明的玻璃态物质。

"对于我们来说,那不是回归。那是\textbf{击穿}。"

\chapter{重力的止痛剂}

警报声在那个深夜拉响。

这一次不是单一的盾构机。整个地下城的芯片都在融化。

自动门疯狂开合,仿佛在试图呼吸。空气循环系统泵入了过量的氧气,导致数十人氧中毒。全息广告牌上的女郎停止了微笑,眼球开始疯狂转动,用一种古老的摩斯电码眨着眼睛:

\textbf{重。给我重。给我痛。}

但最可怕的事情发生在医疗区。

那些患有"嗜铁症"的病人——他们的骨骼因为三百年的高辐射环境而发生了金属化增生,被视为畸形和累赘——此刻全部从病床上坐了起来。

他们的眼睛翻白,口中发出不属于人类的嘶鸣。他们的金属化骨骼在皮肤下发出幽蓝的光芒,像是被通了电的灯丝。

"他们变成了\textbf{天线}。"陆沉冲进医疗区时,立刻明白了发生什么,"那些飞升者的信号……正在通过他们的金属骨骼接收。"

一名嗜铁症患者突然转向陆沉。她的嘴巴张开,发出的却是另一个人的声音——苍老、疲惫、充满了难以言喻的渴望:

\begin{ghostbox}
    "陆沉……我是林溯……"\\
    "告诉他们……我们不是要入侵……"\\
    "我们只是……想落地……"
\end{ghostbox}

陆沉的瞳孔骤然收缩。林溯。那是三百年前带领飞升的林渡的父亲。他也飞升了。

"林溯先生,"陆沉尽量保持冷静,"你们现在的状态——"

\begin{ghostbox}
    "我们在溶解……"\\
    "光速世界里没有时间……瞬间即永恒……"\\
    "我们的意识像墨水滴进大海……正在被稀释……"\\
    "我们需要锚点……需要重力……需要\textbf{痛}……"
\end{ghostbox}

话音未落,那名嗜铁症患者的身体突然剧烈痉挛。她的七窍流出了银色的液体——那是脑组织在高能数据流冲刷下溶解的产物。

林溯的意识太过庞大,人类的大脑根本承载不了。

"好窄……好挤……"患者的嘴里发出最后一声呻吟,然后彻底倒下。

陆沉看着这一幕,握紧了拳头。"他们想回来做人,但他们已经忘了怎么做人。"

老赵跌跌撞撞地跑过来:"怎么办?整个地下城的嗜铁症患者都在发疯!"

陆沉转过身,目光落在医疗区角落里那些蜷缩的、被视为废物的病人身上。他们的金属化骨骼在皮肤下闪烁着诡异的蓝光。

"也许……他们不是废物。"陆沉低声说,"也许他们是\textbf{保险丝}。"

% ==========================================
% 第二部
% ==========================================
\part[介电击穿]{介电击穿\\{\normalsize\textit{Dielectric Breakdown}}}

\chapter{梯级先导}

为了阻止这场灾难,陆沉决定去地表。

不是去谈判——飞升者已经没有谈判的能力了。他们的意识正在以指数级速度退相干,再过几个月,二十亿个独立的"我"就会融化成一团无差别的能量噪音。

他们必须在那之前落地。否则,他们会变成宇宙中最悲哀的存在:一团永恒却无意识的背景辐射。

陆沉启用了"盲视"系统——一种不直接通过光学镜头,而是通过引力波侧影来观察目标的雷达。因为直视那些高能意识体,会直接烧毁视网膜和视神经。

在地表那座黑色的方尖碑前,陆沉停下了脚步。

三百年了,这座碑依然矗立。强相互作用力材料让它成为了时间长河中唯一的礁石。碑体内部还在循环着那段古老的信息:\textit{"外面风很大。但也请你等待。我们会回来的。"}

现在,他们真的回来了。只不过是以一种最惨烈的方式。

方尖碑周围,光辉如雨。那些光点不断撞击着地面,每一次撞击都伴随着数千个意识的尖叫。

陆沉打开了广域广播,用只有纯能量体能听懂的磁场波段喊话:"我是陆沉。我代表'石之民'。我收到了你们的信号。告诉我,你们需要什么?"

光雨停滞了片刻。随后,一个声音直接在陆沉的颅骨内共振。

\begin{ghostbox}
    \textit{"陆沉……我们……好冷……"}
\end{ghostbox}

是林渡。那个三百年前带领人类飞升的女人。现在的她,只是一团破碎的算法集合,声音里充满了静电噪音。

"你们不是去当'引雷针'吗?"陆沉问,"你们吸收到能量了吗?"

\begin{ghostbox}
    \textit{"吸收了……太多了……"}\\
    \textit{"高维空间的能量……比我们想象的要丰沛得多……"}\\
    \textit{"我们变成了一颗过度充电的电池……"}\\
    \textit{"如果不释放……我们会爆炸……"}
\end{ghostbox}

陆沉沉默了。他突然理解了这种悲哀。他们去当引雷针,确实吸收到了能量——但那能量太过庞大,反而把他们撑爆了。

"你们想把能量释放到哪里?"

\begin{ghostbox}
    \textit{"地核……"}\\
    \textit{"费城的洞还在……那是超导通道……"}\\
    \textit{"如果我们能把能量注入地核……"}\\
    \textit{"也许能重启发电机效应……让磁场回来……"}
\end{ghostbox}

陆沉的心跳加速了。

三百年前,地核失速,磁场熄火。这是一切灾难的根源。如果飞升者携带的能量能够注入地核,重新点燃那台沉睡的发电机……

"但你们现在的方式会烧毁一切。"陆沉说,"你们的电弧太强了,地壳承受不住。"

\begin{ghostbox}
    \textit{"所以我们需要……一根引雷针……"}\\
    \textit{"一根能把我们引导进地核的……导线……"}\\
    \textit{"方尖碑……它的外壳是绝缘体……内核是良导体……"}\\
    \textit{"如果能剥开它的外壳……连接到地核……"}
\end{ghostbox}

陆沉低头看着脚下的方尖碑。他突然意识到,这座碑从一开始就不只是一块墓碑。

它是一根\textbf{插头}。

\chapter{接触电阻}

回到地下城,陆沉提出了一个疯狂的计划。

"你要把方尖碑改成什么?"老赵以为自己听错了。

"\textbf{地核导流管}。"陆沉在图纸上画出一个复杂的结构,"方尖碑的外壳是强相互作用力材料——绝缘体。但它的内核是超导物质。我要剥开外壳,把内核连接到深埋地下的地热管道,一直通到地幔边缘。"

"然后呢?"

"然后让飞升者通过这根'导线',把他们携带的能量精准地注入地核。"陆沉的眼神狂热,"相当于给地核做一次\textbf{心脏除颤}。用电击重启它。"

老赵的脸色发白:"这需要穿越几千米的岩层……还有接触电阻的问题。如果导流管和地幔的接触不够紧密,巨大的电流会把接触点熔化!"

"我知道。"陆沉的声音很平静,"所以我需要填充物。一种能在高温高压下保持液态、同时又能完美导电的填充物。"

他转过身,目光落在医疗区的方向。

"嗜铁症患者的血液里,充满了金属离子。他们的骨骼已经高度金属化。"

老赵的瞳孔骤然收缩。"你不是想……"

"我不是想。"陆沉打断他,"是他们\textbf{自己}想。"

他打开了一段录像。那是医疗区的监控画面。

在画面里,那些嗜铁症患者正在做一件不可思议的事——他们在\textbf{排队}。他们拖着沉重的金属化肢体,一个接一个地走向地下城的出口。他们的眼睛里没有恐惧,只有一种近乎虔诚的平静。

"他们听到了。"陆沉低声说,"他们听到了飞升者的呼唤。那些金属骨骼让他们变成了天线,也让他们理解了飞升者的痛苦。"

老赵颤抖着问:"他们要去做什么?"

"他们要去当\textbf{保险丝}。"陆沉闭上眼睛,"导流管的接触电阻过大,需要液态金属来填充。他们会跳进导流槽,让自己的血肉在高压下气化,留下的金属骨骼会融化成完美的导电浆液。"

"这是自杀!"

"这是\textbf{渡}。"陆沉睁开眼,目光坚定,"三百年前,林渡带着飞升者离开。现在,他们要渡回来。而这些被视为废物的病人,将成为渡口的\textbf{桥墩}。"

% ==========================================
% 第三部
% ==========================================
\part[金属骨痂]{金属骨痂\\{\normalsize\textit{The Metal Callus}}}

\chapter{方尖碑手术}

\scene{地表,方尖碑基座}{磁场纪元315年}

工程开始了。

这是人类历史上最疯狂的手术——给一座碑做开颅手术,给一颗行星做心脏搭桥。

陆沉带领着"石之民"的工程队,在地表进行着自杀式的作业。没有磁场保护,宇宙射线像雨点一样倾泻而下。每工作一小时,就有人倒下。但没有人后退。

方尖碑的外壳被一层层剥开。那些强相互作用力材料在高温切割下发出刺耳的尖啸,仿佛这座碑本身也在喊疼。

当内核终于显露出来时,所有人都屏住了呼吸。

那是一根直径三米的银色柱体,表面流动着水银般的光泽。超导物质。在常温下就能实现零电阻的奇迹材料。三百年前,林渡用飞升者的残余能量铸造了这根"内核",等待着今天。

"开始铺设导流管!"陆沉下令。

巨大的钻头刺入地壳。工程机器人沿着预设的路线,将超导管道一节一节地送入岩层深处。目标:地幔边缘,那个曾经沸腾着液态铁镍的地方。

\textbf{七十二小时后。}

导流管铺设完成。但问题来了。

"接触电阻太大了!"工程师的声音充满绝望,"管道末端和地幔的接触面积不够!如果强行通电,接触点会瞬间熔化,整个管道会报废!"

陆沉看着监控屏上那个微小的缝隙。那是导流管末端和地幔之间的空隙,只有几厘米宽。但就是这几厘米,足以让一切功亏一篑。

"需要填充物。"陆沉的声音很轻,"液态金属。"

话音刚落,地下城的出口传来了脚步声。

那些嗜铁症患者来了。

他们拖着沉重的身体,一步一步走向导流槽。他们的眼睛里没有恐惧,只有平静。三百年来,他们被视为畸形、累赘、进化的失败品。今天,他们终于找到了自己存在的意义。

领头的是一个年轻的女孩,看起来不过十五六岁。她的手臂已经完全金属化,在阳光下闪烁着银灰色的光芒。

"我叫银枝。"她对陆沉说,声音平静得不像一个即将赴死的人,"我们听到了天上那些人的声音。他们在哭。他们想回家。"

陆沉张了张嘴,却说不出话。

"我们的骨头里流着铁。"银枝微微一笑,"我们生来就是桥。"

她转过身,带领着队伍走向导流槽。

陆沉伸出手,想要阻止,却被老赵拉住了。

"让他们去。"老赵的声音沙哑,泪水流过他布满皱纹的脸,"这是他们的选择。"

导流槽打开了。

银枝最后看了一眼天空——那片正在被闪电撕裂的天空——然后纵身跳了下去。

身后,数百名嗜铁症患者鱼贯而入。

在高温高压下,他们的血肉瞬间气化,只留下金属化的骨骼。那些骨骼在极端温度下融化,变成流动的银色液体,完美地填满了导流管和地幔之间的每一丝缝隙。

接触电阻归零。

\chapter{行星充电}

\scene{时间:同步}{地点:全球}

陆沉站在方尖碑旁,将粗大的神经导管刺入了自己的脊椎。

没有想象中的剧痛,只有一种冰冷的麻木。那是人类神经被更高维度的能量接管的感觉。

"我会成为开关。"他对老赵说,"我的神经系统会控制能量流入的速度。太快会炸毁地核,太慢飞升者会在抵达前就消散。"

"你会死的。"老赵哽咽着说。

"不,我会变成别的东西。"

陆沉深吸一口气,然后对着天空怒吼:

"来吧!"

\textbf{嗡——}

那道跨越三百光年的电弧,终于有了精准的落点。

无数道闪电汇聚成一道光柱,劈中了方尖碑的顶端。但这一次,没有爆炸,没有毁灭。能量沿着超导内核倾泻而下,穿过数千米的导流管,抵达由金属骨骼铸成的接触层,最终注入沉睡的地核。

光柱持续了九十天。

九十天里,二十亿飞升者的意识和能量,像洪水一样涌入地球。他们不再是独立的个体——那些"我"的边界早已模糊。他们融合成了一股纯粹的、炽热的生命力,注入那颗冰冷的铁镍心脏。

地核开始升温。液态铁镍重新流动。热对流启动。角动量开始累积。

\textbf{第九十一天。}

一道极光在南极洲的天空亮起。

那是三百年来,地球上第一道\textbf{自然极光}。

磁场回来了。

% ==========================================
% 终章
% ==========================================
\chapter*{终章:琥珀里的旧梦}
\addcontentsline{toc}{chapter}{终章:琥珀里的旧梦}
\markboth{终章:琥珀里的旧梦}{终章:琥珀里的旧梦}

\scene{地表,方尖碑基座}{磁场纪元315年,尾声}

风暴停息了。

天空变得干净而黑暗。那些乱码般的极光消失了,取而代之的是久违的、冰冷的星空。偶尔,一道淡绿色的自然极光会在极地上空舞动,像是新生儿的第一次呼吸。

方尖碑矗立在荒原上,表面流动着一层仿佛水银般的银色光辉。它不再是一块死石头,它变成了一个巨大的、跳动的脏器——连接着天上的"光"与地下的"石"。

而在方尖碑的基座上,坐着一个人。或者说,一尊雕像。

救援队赶到时,老赵试图唤醒陆沉。"陆沉?结束了吗?"

那尊"雕像"缓缓转过头。动作僵硬,伴随着金属摩擦的轻响。他的双眼已经没有了瞳孔,只剩下两点幽蓝的磁弧,深不见底。

"结束了。"陆沉的声音不再通过空气震动传播,而是通过磁场直接在老赵的脑海中响起。那种声音冷硬、清晰,没有任何感情色彩。

"飞升者呢?"老赵问。

陆沉抬起那只已经完全金属化的手,指了指脚下的大地。

"在里面。二十亿个意识,融进了地核。"他顿了顿,"他们不再是独立的个体了。他们变成了地球的一部分。变成了磁场的一部分。变成了……热量。"

老赵沉默了。

"他们死了?"

"不,他们\textbf{活}了。"陆沉的声音里第一次出现了一丝温度,"作为个体,他们消失了。但作为能量,他们会永远在地核里燃烧。每一道极光,都是他们的呼吸。每一次地磁脉冲,都是他们的心跳。"

老赵看着陆沉那半人半机器的脸,泪水流了下来。"那你呢?你还要在这里坐多久?"

陆沉没有回答。他现在的逻辑电路里,已经没有"多久"这个概念了。

他只是调整了一下坐姿,让自己与方尖碑的磁场连接得更紧密一些。他是地核与地表之间的桥梁,是新生磁场的稳压器。只要地球还在转动,他就要坐在这里。

"我还有一件事要做。"陆沉突然说。

他伸出手,从方尖碑的基座旁边挖出了一个小小的金属盒子。盒子的表面刻着繁复的花纹,那是三百年前的工艺。

"这是什么?"老赵问。

"琥珀。"陆沉轻轻打开盒子。

里面躺着一块透明的晶体。晶体中央,封存着一小片组织样本——那是一块人类的皮肤,还带着几根毛发。

"这是旧人类的化石。"陆沉说,"三百年前,在飞升大分裂的时候,我的曾祖父从一对死去的情侣身上取下了这块样本。他说,无论我们变成什么,都要记住我们曾经是什么。"

他小心翼翼地把盒子重新埋进方尖碑的基座。

"有一天,也许我们会需要它。"陆沉低声说,"当我们忘记'痛'是什么感觉的时候……当我们忘记'爱'是什么意思的时候……它会提醒我们。"

老赵默默地看着这一切。他知道,眼前的这个人——如果还能称之为人——已经不再属于人类了。但他也知道,正是这个不再是人的存在,拯救了全人类。

"我该回去了。"老赵转过身,"地下城还等着重建。"

"老赵。"陆沉的声音在他脑海中响起。

"嗯?"

"告诉他们,碳基时代结束了。"陆沉的目光穿透了老赵的后背,"为了在这个宇宙活下去,下一代必须学会拥抱铁与磁。必须长出……\textbf{磁骨}。"

老赵没有回头。他只是挥了挥手,然后一步一步走向地下城的入口。

在他身后,那尊金属雕像重新闭上了眼睛,与方尖碑融为一体。

地核深处,二十亿个曾经是人的能量,正在缓缓旋转。

新的发电机,开始运转了。

\vspace{2em}

\begin{techbox}{进化档案:Homo Ferrum(铁血人)}
    \textbf{样本编号}:001(陆沉)\\[0.5em]
    \textbf{特征}:
    \begin{itemize}[leftmargin=*, itemsep=2pt]
        \item 碳基-金属基混合结构
        \item 神经系统量子化,与地磁场共振
        \item 能够直接感知并操控电磁场
        \item 能量来源:地核热能 / 地磁脉冲
    \end{itemize}
    \textbf{状态}:存活。作为地核能量的稳压器,永久驻守于方尖碑基座。\\[0.5em]
    \textbf{附注}:基座下方封存有旧人类组织样本(代号"琥珀"),用途待定。
\end{techbox}

\vfill
\begin{center}
    \rule{0.3\textwidth}{0.5pt}\\[1em]
    {\large 第二部 · 完}
\end{center}

\end{document}