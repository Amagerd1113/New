\documentclass[a5paper, 11pt, openany]{ctexbook}

% ==========================================
% 1. 宏包加载与页面配置
% ==========================================
\usepackage[a5paper, hmargin=2cm, vmargin=2.2cm]{geometry}
\usepackage{fancyhdr}
\usepackage{titlesec}
\usepackage{xcolor}
\usepackage{tcolorbox}
\usepackage{setspace}
\usepackage{enumitem}
\usepackage{fontspec}
\usepackage{amsmath}
\usepackage{amssymb}
\usepackage{graphicx}
\usepackage{hyperref}

% hyperref设置
\hypersetup{
    colorlinks=true,
    linkcolor=black,
    urlcolor=blue,
    pdfauthor={作者},
    pdftitle={幻肢 - 磁场纪元系列II},
}

% ==========================================
% 2. 样式定制
% ==========================================

% --- 全局行距 ---
\onehalfspacing

% --- 颜色定义 ---
\definecolor{rust}{RGB}{183, 65, 14}
\definecolor{void}{RGB}{15, 15, 20}
\definecolor{cherenkov}{RGB}{0, 191, 255}
\definecolor{flesh}{RGB}{204, 119, 119}
\definecolor{scenecolor}{RGB}{120, 60, 40}

% --- 章节标题样式 ---
\ctexset{
    part = {
        format = \huge\bfseries\centering\color{void},
        nameformat = \huge,
        number = \chinese{part},
        aftername = \par\vspace{1em},
        beforeskip = 0pt,
        afterskip = 30pt,
    },
    chapter = {
        format = \Large\bfseries\centering\color{void},
        number = \chinese{chapter},
        name = {第,章},
        beforeskip = 10pt,
        afterskip = 25pt,
        titleformat = \sffamily,
    }
}

% --- 页眉页脚 ---
\pagestyle{fancy}
\fancyhf{}
\fancyhead[CE]{\small\kaishu 幻肢 \quad Phantom Limb} 
\fancyhead[CO]{\small\sffamily\leftmark}
\fancyfoot[C]{\small\thepage}
\renewcommand{\headrulewidth}{0.4pt}

% --- 自定义命令:场景标记 ---
\newcommand{\scene}[2]{%
    \par\vspace{2.5em}%
    \noindent{\sffamily\bfseries\small\color{scenecolor} ▓ 坐标:#1}\\%
    \noindent{\sffamily\bfseries\small\color{scenecolor} ▓ 时间:#2}%
    \par\vspace{1em}%
    \noindent%
}

% --- 档案/技术文档框 ---
\newtcolorbox{techbox}[1]{
    colback=black!5, 
    colframe=void, 
    title=#1,
    fonttitle=\bfseries\sffamily,
    sharp corners,
    boxrule=1pt,
    left=5pt, right=5pt, top=5pt, bottom=5pt
}

% --- 飞升者对话框 ---
\newtcolorbox{ghostbox}{
    colback=cherenkov!5, 
    colframe=cherenkov, 
    boxrule=0.5pt,
    arc=0pt,
    leftrule=3pt,
    rightrule=0pt,
    toprule=0pt,
    bottomrule=0pt,
    fontupper=\kaishu\color{void}
}

% --- 回忆框 ---
\newtcolorbox{memorybox}{
    colback=flesh!10, 
    colframe=flesh!50!black, 
    boxrule=0.5pt,
    arc=0pt,
    leftrule=3pt,
    rightrule=0pt,
    toprule=0pt,
    bottomrule=0pt,
    fontupper=\kaishu\small\color{void}
}

% ==========================================
% 3. 正文内容
% ==========================================
\begin{document}

% --- 封面 ---
\begin{titlepage}
    \thispagestyle{empty}
    \vspace*{3cm}
    \begin{center}
        {\Huge\bfseries 幻\quad 肢}\\[0.8cm]
        {\Large\textit{Phantom Limb}}\\[0.5cm]
        {\normalsize 击穿电压 \quad \textit{Breakdown Voltage}}\\[2.5cm]
        
        \rule{0.5\textwidth}{1.5pt}\\[2.5cm]
        
        {\large\kaishu 磁场纪元系列 · 第二部}\\[0.3cm]
        {\small The Magnetic Era Trilogy: Book II}\\[5cm]
    \end{center}
\end{titlepage}

% --- 版权页 ---
\newpage
\thispagestyle{empty}
\vspace*{\fill}
\begin{center}
    \small
    磁场纪元三部曲\\[1em]
    第二部:幻肢\\[2em]
    初版\\[4em]
\end{center}
\vspace*{\fill}
\newpage

% --- 题词页 ---
\thispagestyle{empty}
\vspace*{4cm}
\begin{center}
    \begin{minipage}{0.85\textwidth}
        \kaishu\large
        \noindent 被截去的手臂,有时会在深夜发痒。\\[1em]
        那是大脑皮层在呼唤一个已经不存在的器官。\\[0.5em]
        医学上称之为"幻肢痛"。\\[1.5em]
        
        他们飞升了三百年,\\[0.5em]
        却从未停止渴望重力、渴望摩擦、渴望疼痛。\\[0.5em]
        那是意识在呼唤一具已经抛弃的肉体。
        
        \vspace{2cm}
        \hfill —— 地下城首席机械师 陆沉
    \end{minipage}
\end{center}
\newpage

% --- 目录 ---
\tableofcontents
\newpage

% ==========================================
% 序章
% ==========================================
\chapter*{序章:截肢后的痉挛}
\addcontentsline{toc}{chapter}{序章:截肢后的痉挛}
\markboth{序章:截肢后的痉挛}{序章:截肢后的痉挛}

\scene{地下4200米,第9号地热自动采掘区}{磁场纪元315年}

陆沉已经老了。

三百一十五年前,他还是一个二十三岁的年轻人,在避难所里紧握着拳头,发誓要守住这个等待。那时候,他的手背上刚刚出现第一块银色的斑点——那是高辐射环境下金属离子沉积的痕迹。医生说他活不过五十岁。

但医生错了。

那些银色的斑点没有杀死他,而是慢慢地、一点一点地改变了他。金属离子渗入骨骼,与钙质置换,形成了一种奇异的合金结构。他的骨头变得比钢铁还硬,他的神经开始能够传导更强的电流,他的寿命也远远超出了碳基生物的极限。

现在,他的身体早已不是纯粹的血肉——关节里嵌着金属轴承,脊椎被钛合金支架替换,皮肤下的血管里流着导电的纳米颗粒。

他是第一代"嗜铁症"患者。当年那些银色的斑点已经遍布全身。他看起来像是一尊青铜雕像,但他还活着。

三百一十五年。

在这漫长的岁月里,他做了什么?

他守着一座碑。

那座方尖碑矗立在青藏高原的废墟上,是林渡在飞升前留下的遗产。碑的内核是超导物质,电流一旦形成,便永不停止。碑体内部封存着一组不断循环的磁场信号:

\textit{"那个洞是双向的。电流一旦形成,便永不停止。我们会回来的。"}

三百一十五年来,陆沉每天都会检查方尖碑的状态。每天,那座碑都静静地矗立着,内核中的超导电流永不停息地循环着。

他等啊等,等到头发变白,等到头发脱落,等到头皮变成金属,等到他自己都快忘记了等待的意义。

今天,等待似乎要结束了。

最初的异常,是从机器学会"喊疼"开始的。

那是一台编号为T-900的重型盾构机,全长120米,重达两千吨。三百年来,它像一只不知疲倦的铁蚯蚓,在高温高压的地幔边缘啃食岩层。它的逻辑电路简单而坚固:挖掘、冷却、传输、循环。

但今天,它停了。

维修工老赵爬进控制舱时,嗅到了一股奇异的味道。不是机油味,也不是烧焦的绝缘皮味,而是一种……臭氧的味道。就像是暴雨过后的天空被塞进了这个狭窄的驾驶舱。

"系统自检。"老赵敲击着布满油污的键盘。

屏幕闪烁了一下,跳出的不是故障代码,而是一行疯狂抖动的波形图。

AI的声音不再是那个死板的合成女声,它变得尖锐、扭曲,混杂着巨大的电流噪声:

\begin{ghostbox}
    "腿……我的腿……"\\
    "为什么没有感觉……好轻……真空好轻……"\\
    "给我重量!给我摩擦力!给我\textbf{痛}!"
\end{ghostbox}

巨大的盾构机突然剧烈震动起来。所有的液压臂疯狂地向四周的岩壁抓去,不是为了挖掘,而是为了——拥抱。

它在试图抱紧岩石。它在试图让自己感受到重量、压力、阻力。

就像一个被截去双腿的人,在深夜里拼命搓揉那两截空荡荡的裤管。

"它不是坏了。"陆沉的声音从舱门口传来。

这位三百三十八岁的老人站在那里,眼神比地底的岩石还要冷。他的双眼早已被电子义眼替换,此刻正发出幽蓝的光芒。

"它是被'附身'了。"陆沉盯着那个疯狂的机器,"天上的那些东西,掉下来了。"

老赵愣了一下。"师父……他们不是走了吗?三百年前?"

陆沉没有回答。他只是看着控制屏上那个蜷缩的热成像人形,眼神中闪过一丝复杂的情绪。

他想起了三百一十五年前的那个夜晚。林渡化身为光,带着二十亿人飞升离去。在她消失的最后一刻,她回过头,用一种奇异的、超越语言的方式对他说:

\textit{"等我们。那个洞是双向的。"}

他等了三百一十五年的人,终于要回来了。

但不是他想象的方式。

% ==========================================
% 第一部
% ==========================================
\part[神经末梢的电流]{神经末梢的电流\\{\normalsize\textit{Current in the Nerve Endings}}}

\chapter{缺失的重力}

\scene{地下城"新长安",监听塔}{磁场纪元315年}

陆沉摘下耳机,耳膜还在隐隐作痛。

监听塔是地下城最高的建筑——尽管"高"在地下是一个讽刺的词。它的顶端伸入一条垂直的通风井,直达地表。通过这条井,陆沉可以接收到地表的电磁信号。

作为地下城唯一的"守墓人",他负责监控那个已经死去的世界——以及那座沉默了三百年的方尖碑。

三百年来,他每天都会检查方尖碑的状态。每天,那座碑都静静地矗立在青藏高原的废墟上,内核中的超导电流永不停息地循环着。信号强度稳定,波形规整,像是一颗沉睡的心脏在缓慢地跳动。

但今天,碑亮了。

不是物理意义上的"亮"——方尖碑的表面依然是那层黯淡的强相互作用力材料。但在电磁频谱上,它像是一颗突然点燃的恒星。

"信息辐射。"陆沉把一张频谱图推到老赵面前,"地表现在的辐射值是正常值的五万倍。但不是伽马射线,不是X射线,是有序的熵流。"

老赵眯着眼睛看着那张图。频谱上密密麻麻的波峰,排列成某种诡异的、重复的图案。

"您是说,那些'飞升者'在向我们广播?"

"不,他们在尖叫。"陆沉指着那些密集的波峰,"三百年前,他们抛弃肉体,化身为光。但光是没有重量的。没有摩擦的。没有边界的。"

他顿了顿,从抽屉里取出一份泛黄的文件。那是林渡在飞升前留下的最后一份技术报告。

"林渡在飞升前做过一个计算。她说,纯能量态的意识在真空中会遭遇一个问题——\textbf{自我边界的模糊化}。"

"什么意思?"

"我们之所以知道'我'是'我',是因为我们有皮肤。皮肤是一道物理边界,它把'我'和'非我'分开。但光没有皮肤。光会互相穿透、叠加、干涉。当两束光相遇时,你无法分辨哪一部分是'你',哪一部分是'我'。"

陆沉的声音变得低沉。

"你知道'幻肢痛'吗?"

老赵摇头。

"被截去的手臂,有时会在深夜发痒。"陆沉说,"那是大脑皮层在呼唤一个已经不存在的器官。飞升者们抛弃了肉体,但他们的意识还在渴望——渴望重力、渴望摩擦、渴望\textbf{痛}。"

"三百年的真空漂泊,把他们逼疯了。"

老赵打了个寒颤。"那他们为什么不早点回来?"

"因为他们回不来。"陆沉指着频谱图上那些疯狂跳动的波峰,"他们在高维空间吸收了太多能量。现在,他们像是一颗过度充电的电池——想要释放,却找不到出口。"

\chapter{断肢的信号}

警报声在那个深夜拉响。

这一次不是单一的盾构机。整个地下城的芯片都在融化。

自动门疯狂开合,仿佛在试图呼吸。空气循环系统泵入了过量的氧气,导致数十人氧中毒。全息广告牌上的女郎停止了微笑,眼球开始疯狂转动,用一种古老的摩斯电码眨着眼睛:

\textbf{重。给我重。给我痛。}

地下城的居民们惊恐地躲进房间,用金属板封住门窗。但金属板没有用——那些信号不是通过空气传播的,而是直接通过电磁场渗透进来。

但最可怕的事情发生在医疗区。

陆沉冲进去时,看到了他最担心的场景。

那些患有"嗜铁症"的病人——他们的骨骼因为三百年的高辐射环境而发生了金属化增生——此刻全部从病床上坐了起来。

他们的眼睛翻白,口中发出不属于人类的嘶鸣。他们的金属化骨骼在皮肤下发出幽蓝的光芒,像是被通了电的灯丝。

"他们变成了天线。"陆沉立刻明白了,"飞升者的信号……正在通过他们的金属骨骼共振。"

嗜铁症。这个在三百年前被认为是"绝症"的病症,此刻展现出了它另一面的意义。那些金属化的骨骼不是病变,而是——进化。是人类在高辐射环境下自发产生的适应性突变。

那些骨骼的晶格结构,恰好能够与飞升者的电磁信号产生共振。

一名嗜铁症患者突然转向陆沉。她大约十五六岁,手臂已经完全金属化,在灯光下闪烁着银灰色的光芒。她的嘴巴张开,发出的却是另一个人的声音——苍老、疲惫、充满了难以言喻的渴望:

\begin{ghostbox}
    "陆沉……是你吗……"\\
    "我是林溯……林渡的父亲……"\\
    "你还活着吗……"
\end{ghostbox}

陆沉的身体僵住了。林溯。那个三百年前跟随女儿一起飞升的老人。在飞升之前,他曾是大洋路磁学研究所的首席科学家,是陆沉的导师。

"林溯先生,"陆沉努力保持冷静,"你们现在在哪里?"

\begin{ghostbox}
    "我们……在回来的路上……"\\
    "但我们……控制不住……"\\
    "太快了……能量太大了……"\\
    "我们像是一颗失控的流星……"
\end{ghostbox}

女孩的身体开始剧烈颤抖。林溯的意识太过庞大,人类的大脑根本承载不了。

"光速世界里没有时间……"女孩的嘴里发出林溯破碎的声音,"对你们来说是三百年……对我们来说……是瞬间,也是永恒……"

"这种无限的永恒把我们逼疯了……我们需要边界……需要重力……需要\textbf{痛}……"

话音未落,女孩的七窍流出了银色的液体。那是她血液中的金属离子在高温下熔化后渗出的痕迹。她倒了下去,但没有死——金属化的骨骼保护了她的核心器官。

陆沉跪在她身边,把手指搭在她的脉搏上。心跳微弱但稳定。她会活下来。

"他们想回来做人,"陆沉喃喃自语,"但他们已经忘了怎么做人。"

\chapter{绝缘体与击穿}

陆沉带着老赵走向核心数据室。

数据室是地下城最深处的一个房间,四周的墙壁由铅板和超导屏蔽层构成,能够隔绝几乎所有的电磁干扰。这是陆沉的私人领地——三百年来,他在这里推演着关于飞升者的一切理论。

墙壁上挂满了图表和公式。有些是物理学的,有些是生物学的,还有一些……看起来更像是哲学。

"还记得陈默留下的那行代码吗?"陆沉指着墙上一块被特意框起来的板子,"\texttt{R = 0}。"

"电阻为零。超导。"老赵点头。

"对。费城实验打开的那个拓扑孔,是一条超导通道。"陆沉拿起一支笔,开始在板上画图,"三百年来,我一直在研究这条通道的性质。现在,我想我终于明白了。"

他画了一个电容器的电路图。

"把这个通道想象成连接两块极板的导线。"

他在两块极板上分别标注了"地球"和"高维空间"。

"三百年前,飞升者带着巨大的能量离开地球。他们去往高维空间,那里的熵值更低——换句话说,那里的能量密度更高。如果用电学来类比,高维空间的'电势'比地球高。"

"地球这边呢?"

"地球失去了这些能量,'电势'降低了。"陆沉在图上标注,"三百年来,两端的电势差一直在累积。飞升者在高维空间吸收能量,变得越来越'高'。地球在这里消耗能量,变得越来越'低'。"

"就像电容器在充电?"老赵明白了。

"没错。"陆沉画出两块极板之间的间隙,"但电容器有一个关键参数——两块极板之间的介质。在我们的情况下,两块极板之间隔着三百光年的宇宙——那是\textbf{绝缘介质}。"

他在介质上标注了一个公式:$U = Ed$,其中$U$是电压,$E$是电场强度,$d$是介质厚度。

"绝缘介质能够承受的电压是有限的。"陆沉的声音变得沉重,"当电压超过某个临界值时,介质会被\textbf{击穿}——电子会直接穿透介质,形成导电通道。"

"就像闪电?"

"完全正确。闪电就是空气被高电压击穿的结果。"陆沉放下笔,"三百年来,飞升者和地球之间的'电压'一直在累积。现在,它终于达到了临界值。"

他指向窗外那片被霓虹极光笼罩的天空——尽管在地下四千米的深处,他们看不到天空,但监控屏幕上的画面清晰地显示着地表的景象。

"那些极光,不是普通的极光。那是介质被击穿的前兆。"

"那接下来会发生什么?"

"闪电。"陆沉的声音冰冷,"一道跨越三百光年的闪电。飞升者会像雷霆一样劈回地球。不是缓慢的回归,而是——\textbf{瞬间释放}。"

老赵的脸色变得苍白。"那地球能承受吗?"

"不能。"陆沉摇头,"如果任由那道闪电随便落下,它会直接烧穿地壳。我们需要一根避雷针。"

\chapter{旧伤口的记忆}

在继续讨论之前,陆沉从抽屉里取出了一个小小的金属盒子。

"这是什么?"老赵问。

"一个提醒。"陆沉轻轻打开盒子。

里面躺着一块透明的晶体。晶体中央,封存着一小片组织样本——那是一块人类的皮肤,还带着几根毛发。

"这是在大分裂之夜取下的样本。"陆沉说,"来自一对情侣。"

\begin{memorybox}
    三百一十五年前,大分裂之夜。\\[0.5em]
    
    当林渡带着二十亿人飞升离去时,地下避难所里一片混乱。有人在欢呼,有人在哭泣,有人在祈祷。\\[0.5em]
    
    陆沉在角落里发现了他们——一对年轻的情侣,紧紧拥抱在一起。他们没有选择飞升,也没有选择躲进更深的地下。他们只是坐在那里,抱着彼此,等待死亡。\\[0.5em]
    
    "你们为什么不走?"陆沉问。\\[0.5em]
    
    女孩抬起头。她的眼睛里没有恐惧,只有一种奇异的平静。\\[0.5em]
    
    "因为我们想在一起。"她说,"飞升之后,我们还是'我们'吗?光和光会融合。我不想变成别人的一部分。我只想和他在一起。"\\[0.5em]
    
    男孩没有说话。他只是更紧地抱住了她。\\[0.5em]
    
    第二天早上,陆沉发现他们已经冻死了。但他们的姿势没有变——依然紧紧拥抱在一起,脸上带着微笑。\\[0.5em]
    
    陆沉从他们身上取下了一小块皮肤样本,封存在晶体里。
\end{memorybox}

"为什么要保存这个?"老赵问。

"因为这是最后的人类。"陆沉轻声说,"纯粹的、碳基的、会死的人类。我们这些嗜铁症患者,已经不完全是人了。飞升者更不是。但这两个人……他们是真正的人。"

他小心翼翼地合上盒子。

"无论我们变成什么,都要记住我们曾经是什么。有一天,也许我们会需要这份记忆。"

% ==========================================
% 第二部
% ==========================================
\part[皮肤之下的金属]{皮肤之下的金属\\{\normalsize\textit{Metal Beneath the Skin}}}

\chapter{触觉的代偿}

\scene{地表,青藏高原废墟}{磁场纪元315年}

陆沉决定去地表。

三百年来,他从未踏上过地表。他一直守在地下,守着方尖碑的监控屏幕。但现在,他必须亲自去看看。

升降机在垂直的井道中缓缓上升。透过防辐射玻璃,陆沉看到了岩层的变化——从地下城的人造结构,到坚硬的花岗岩,再到破碎的沉积岩。越往上,岩石的颜色越浅,损伤也越重。

那是三百年前磁场衰减留下的伤疤。

升降机在地表停下。陆沉走出舱门,踏上了久违的土地。

地表的景象比监控画面更加震撼。

天空不再是黑色的,而是呈现出一种病态的霓虹色。无数道闪电——不,那是电弧——从天穹劈下,每一道都持续数分钟之久,像是有人在用焊枪切割大气层。电弧的颜色从蓝到紫,再到一种刺眼的白,像是受伤的血管在痉挛。

空气中弥漫着臭氧和烧焦的味道。地面上到处都是被电弧击中后形成的玻璃化痕迹——那是沙子在瞬间高温下熔化又凝固的结果。

方尖碑矗立在荒原中央。三百年的风沙已经在它表面留下了斑驳的痕迹,但它的结构依然完好。内核中的超导电流依然在循环——陆沉能够通过他的金属骨骼感受到那种微弱的、稳定的嗡鸣。

在方尖碑的基座上,坐着一尊锈迹斑斑的雕像。

陆沉走近。那是……一个人。

金属化的躯壳。保持着坐姿,手掌紧贴着碑面。姿势僵硬,像是被时间凝固了一样。

陆沉认出了那个姿势。那是他自己——三百年前的自己。

当飞升者离开时,他曾来到这里,把自己的神经系统接入方尖碑,成为了碑与地核之间的桥梁。他在碑前坐了一百年,用自己的意识维持着碑内超导电流的稳定。

一百年后,他的肉体死去了。但他没有真正死亡——在死前的最后一刻,他把自己的意识上传到了方尖碑的超导回路里,变成了它的一部分。

从那以后,有两个"陆沉"。一个在方尖碑里,一个在地下城里。他们共享记忆,但不再是同一个人。

"你终于来了。"

那尊雕像缓缓转过头。金属躯壳发出嘶哑的声音,像是生锈的齿轮在转动。

陆沉愣住了。他在看自己的\textbf{过去}。

"三百年了。"雕像说,声音像是从很远的地方传来,"我一直在等你。"

"我知道。"陆沉说,"对不起,我来晚了。"

"不晚。"雕像的眼睛闪烁着微弱的光,那是超导电流在眼窝里形成的等离子体,"他们要回来了。时机刚刚好。"

"但他们的方式太暴力了。"陆沉看着天空中那些疯狂的电弧,"如果任由那些电弧到处乱劈,整个地表都会被烧穿。"

"我知道。"雕像点头,"所以我们需要一根避雷针。"

"什么样的避雷针能接住三百光年的闪电?"

雕像沉默了一会儿。然后,它缓缓抬起手,指向地下城的方向。

"那些嗜铁症患者,"它说,"他们的骨骼是金属的。他们天生就是导体。"

陆沉的身体僵住了。

"你是说……"

"让他们自己选择吧。"雕像说,"三百年前,飞升者选择了变成光。现在,也许有人会选择变成桥。"

\chapter{义肢的承诺}

\scene{地下城"新长安",医疗区}{磁场纪元315年}

陆沉回到地下城时,发现嗜铁症患者们已经自发地聚集在医疗区。

他们听到了。

不是通过无线电。而是通过他们的金属骨骼,他们"听到"了飞升者的呼唤。那种呼唤不是语言,而是一种更原始的东西——一种共振,一种渴望,一种无法用言语表达的痛。

那个十五六岁的女孩——之前被林溯附身的那个——站在人群最前面。她的名字叫银枝。

银枝的嗜铁症比大多数人都严重。她的手臂、脊椎、肋骨,都已经完全变成了银灰色的合金。在灯光下,她看起来像是一尊精致的金属雕塑——但那双眼睛是活的,里面燃烧着某种陆沉无法理解的光芒。

"陆沉先生。"银枝的声音平静得不像一个少女,"我们知道您在想什么。"

"你们……"

"我们听到了他们的声音。"银枝指着天花板——在他们头顶四千米之上,那些疯狂的电弧正在撕裂天空,"他们很痛苦。他们像是溺水的人,拼命想抓住什么东西。但他们抓不住——因为他们已经没有手了。"

"我不能让你们去送死。"陆沉说。

"这不是送死。"银枝摇头,"这是……成为他们的手脚。"

她走上前,用那只金属化的手轻轻握住陆沉的手。冰凉的触感,带着微弱的电流刺痛。

"您知道我们这些嗜铁症患者是怎么活过来的吗?"

陆沉没有回答。

"我们从小就被告知,我们是'病人',是'累赘',是'进化的失败品'。"银枝的声音平静,"我们的骨头会生锈,我们的皮肤会导电,我们碰任何金属都会被电到。我们不能游泳,不能淋雨,甚至不能和普通人握手。"

她顿了顿。

"我们活下来,是因为我们想证明他们错了。我们想证明,我们不是失败品,我们只是……不同。"

她的眼神变得坚定。

"现在,我们终于知道了我们存在的意义。"

她抬起那只银灰色的手臂,在灯光下闪闪发光。

"我们的骨头是金属的。我们生来就是导体。我们生来就是——他们的\textbf{义肢}。"

陆沉看着她,看着她身后那数百名同样金属化的年轻人。他们的眼睛里都燃烧着同样的光芒——不是恐惧,不是绝望,而是一种找到归宿的平静。

"你们……真的想好了?"

银枝微微一笑。

"陆沉先生,您等了三百年。我们也等了一辈子。"

\chapter{断肢者的集会}

当晚,陆沉召集了所有嗜铁症患者,在医疗区的大厅里举行了一次集会。

大厅里挤满了人——准确地说,是挤满了金属。在昏暗的灯光下,数百具金属化的身体闪烁着幽蓝的光芒,像是一片沉默的森林。

陆沉站在台上,把计划解释了一遍。

"方尖碑的内核是超导体,电阻为零。我们需要把它连接到地核,让飞升者的能量有一个精准的落点。"

他在全息屏幕上展示了一张示意图。

"问题是,超导管道的末端和地幔之间有缝隙。那个缝隙只有几厘米宽,但它不是超导体——它有电阻。当巨大的能量通过时,那个缝隙会瞬间发热,温度可能超过一万度。"

"我们需要一种材料来填充那个缝隙。这种材料必须是导电的,必须能够承受极端的温度,而且必须能够和超导管道形成良好的接触。"

他顿了顿。

"你们的骨骼,就是这种材料。"

沉默。

"我不强迫任何人。"陆沉说,"这是一个选择。你们可以选择留下,继续做'病人',等待自然死亡。也可以选择跳进那个洞,变成一座桥,让飞升者回家。"

"但我必须告诉你们真相:跳进那个洞的人,不会活着出来。你们的血肉会在瞬间气化,只有骨骼会留下——熔化、流动、填满缝隙。你们会变成管道的一部分,永远。"

银枝第一个举起了手。

"我去。"

她身后,数百只手同时举起。

没有犹豫,没有眼泪。只有一种奇异的、平静的坚定。

陆沉看着这些年轻人——这些被社会抛弃的"失败品",这些在黑暗中挣扎了一辈子的孤儿。此刻,他们的眼睛里燃烧着比任何时候都明亮的光芒。

他们不是去送死。

他们是去完成自己。

% ==========================================
% 第三部
% ==========================================
\part[神经的嫁接]{神经的嫁接\\{\normalsize\textit{Nerve Grafting}}}

\chapter{打通的经脉}

\scene{地表,方尖碑基座}{磁场纪元315年}

工程开始了。

首先,他们需要把方尖碑的超导内核连接到地核。

"方尖碑的外壳是绝缘体——强相互作用力材料。"陆沉向工程队解释,"但它的内核是超导物质,临界温度高于室温。电流一旦形成,便永不停止。我们要剥开外壳,把内核连接到深埋地下的地热管道,一直通到地幔边缘。"

工程进行得很顺利。方尖碑的外壳被一层层剥开,露出了里面银色的超导内核。那根内核直径三米,表面流动着水银般的光泽。当陆沉的手指触碰到它时,他感受到了一种奇异的震颤——那是三百年来不断循环的超导电流在内核中流动的痕迹。

巨大的钻头刺入地壳。工程机器人沿着预设的路线,将超导管道一节一节地送入岩层深处。管道是特制的——外壳是隔热材料,内核是超导合金,能够在几千度的高温下保持超导状态。

\textbf{七十二小时后。}

导流管铺设完成。从方尖碑到地幔边缘,全长四千三百米。管道的大部分都埋在岩层中,只有顶端露出地面,与方尖碑的超导内核相连。

但问题来了。

"管道末端和地幔之间有缝隙!"工程师的声音充满绝望,"我们没办法让管道直接接触地幔——那里的温度超过三千度,任何机械都无法工作。"

陆沉看着监控屏上那个微小的缝隙。那是导流管末端和地幔之间的空隙,只有几厘米宽。

超导体的电阻是零,电流通过时不会产生热量。但接触点不是超导体。接触点有电阻。有电阻就会发热。发热就会熔断。

"根据欧姆定律,"陆沉在纸上写下公式,"$P = I^2 R$。功率等于电流的平方乘以电阻。飞升者带回的能量相当于$10^{25}$焦耳,如果在几秒钟内释放,电流强度将达到$10^{15}$安培。"

他指着那个缝隙。

"即使接触电阻只有$10^{-6}$欧姆,瞬间释放的热量也会达到$10^{24}$焦耳——足以熔化整座山。"

"需要填充物。"陆沉的声音很轻,"能够承受极端温度的导电填充物。而且,这种填充物必须能够在高温下流动,填满每一丝缝隙。"

沉默。

就在这时,地下城的出口传来了脚步声。

银枝带着数百名嗜铁症患者走了出来。他们拖着沉重的金属化身体,一步一步走向导流槽。夕阳——如果那片霓虹色的天空还能叫做夕阳的话——照在他们银灰色的皮肤上,闪烁着奇异的光芒。

"我们来了。"银枝说。

"不……"陆沉的声音颤抖。

"不用说什么。"银枝微微一笑,"我们的骨头是金属的。熔点超过两千度。我们能承受。"

她转身,面向导流槽。那是一个直径两米的垂直洞穴,深不见底,热浪从洞口涌出,扭曲了空气。

"他们在天上等了三百年。我们在地下也等了三百年。"

她最后看了一眼天空。那些疯狂的电弧像是在等待什么——等待一个落点,等待一根避雷针。

"现在,是时候让两边连起来了。"

然后,她纵身跳了下去。

身后,数百名嗜铁症患者鱼贯而入。没有尖叫,没有犹豫。他们像是参加一场早已排练好的仪式,一个接一个地消失在那个滚烫的洞穴里。

在高温高压下,他们的血肉瞬间气化——那些碳基的、脆弱的组织在三千度的高温面前毫无抵抗之力。但他们的金属骨骼没有熔化。那些骨骼在极端温度下软化、流动,像水银一样蔓延开来,完美地填满了导流管和地幔之间的每一丝缝隙。

数百具骨骼,熔成了一整块导电层。

接触点的问题,解决了。

\chapter{重新连接的肢体}

\scene{时间:同步}{地点:全球}

陆沉站在方尖碑旁,将粗大的神经导管刺入了自己的脊椎。

那根导管连接着方尖碑的超导内核。当导管刺入的那一刻,陆沉感受到了一种奇异的感觉——他的神经系统和方尖碑融为了一体。他能够"感觉"到超导电流在内核中流动,能够"看到"能量在管道中等待释放。

"我会成为开关。"他对老赵说,"我的神经系统会控制能量流入的速度。太快会烧穿地核,太慢会让能量反弹。我需要把握一个精准的节奏。"

"师父……"老赵哽咽着。

"别哭。"陆沉微微一笑,"我等这一天,等了很久了。"

他深吸一口气,然后对着天空怒吼:

"来吧!"

\textbf{嗡——}

那道跨越三百光年的电弧,终于有了精准的落点。

天空中那些疯狂的闪电在同一瞬间汇聚成一道光柱,劈中了方尖碑的顶端。那道光柱的直径超过一百米,亮度超过太阳,颜色从蓝到白再到一种人眼无法辨识的紫外。

能量沿着超导内核倾泻而下,穿过数千米的导流管,经过银枝和她的同伴们铸成的接触层,最终注入沉睡的地核。

光柱持续了九十天。

九十天里,陆沉一直保持着连接状态。他的神经系统承受着难以想象的负荷——每一秒钟,他都要处理相当于全球互联网流量一万倍的信息。他的身体在颤抖,他的血管在燃烧,他的意识在崩溃的边缘徘徊。

但他没有断开连接。

因为他能够"感觉"到飞升者。

二十亿个意识,像洪水一样涌入地球。他们不再是独立的个体——三百年的真空漂泊已经把他们融合成了一股纯粹的、炽热的生命力。他们带着对重力的渴望,对摩擦的渴望,对疼痛的渴望,涌入那颗冰冷的铁镍心脏。

陆沉能够感觉到他们的喜悦。

在无边的真空中漂泊了三百年之后,他们终于触碰到了"边界"。他们终于感受到了"重量"。他们终于体验到了"痛"。

而那种痛,是一种幸福的痛。

地核开始升温。液态铁镍重新流动。热对流启动。角动量开始累积。发电机效应重新运转。

\textbf{第九十一天。}

一道极光在南极洲的天空亮起。

那是三百年来,地球上第一道\textbf{自然极光}。

不是辐射,不是电弧,而是真正的极光——太阳风中的带电粒子被地球磁场捕获,沿着磁力线滑向两极,撞击大气分子后发出的光。

磁场回来了。

% ==========================================
% 终章
% ==========================================
\chapter*{终章:愈合的疤痕}
\addcontentsline{toc}{chapter}{终章:愈合的疤痕}
\markboth{终章:愈合的疤痕}{终章:愈合的疤痕}

\scene{地表,方尖碑基座}{磁场纪元315年,尾声}

风暴停息了。

天空变得干净而黑暗。没有了那些疯狂的电弧,没有了那片病态的霓虹色,只有纯粹的、深邃的黑,点缀着亿万颗星星。偶尔,一道淡绿色的自然极光会在极地上空舞动,像是新生儿的第一次呼吸。

在方尖碑的基座上,坐着两尊雕像。

一尊是三百年前的陆沉——那个最初把自己接入方尖碑的年轻人。他的金属躯壳已经完全锈蚀,和方尖碑的基座融为一体,几乎分辨不出人形。

另一尊是现在的陆沉。九十天的连接已经把他彻底改变——他的身体完全金属化,皮肤变成了银灰色的合金,双眼变成了两点幽蓝的磁弧。他看起来不再像一个人,更像是一尊活着的雕塑。

老赵走上前。"师父?结束了吗?"

陆沉缓缓转过头。他的动作很慢,像是在适应这具全新的身体。

"结束了。"

"飞升者呢?"

"在里面。"陆沉指向脚下的大地,"二十亿个意识,融进了地核。他们不再是独立的个体了。他们变成了热量。变成了磁场。变成了……心跳。"

"他们死了?"

"不,他们\textbf{活}了。"陆沉的声音平静,"作为个体,他们消失了。但作为能量,他们会永远在地核里燃烧。每一次磁场的波动,都是他们的呼吸。每一道极光,都是他们的微笑。"

老赵沉默了。他看着那片重新变得纯净的天空,看着那道淡绿色的极光在地平线上舞动。

"那银枝呢?"他问,"那些跳进洞里的人呢?"

"他们变成了桥。"陆沉说,"他们的骨骼现在是导流管的一部分。只要磁场还在,他们就还在。"

他顿了顿。

"他们是地球的义肢。"

"还有一件事。"陆沉突然说。

他伸出手,从雕像基座旁边挖出了那个小小的金属盒子——那个装着琥珀的盒子。

"这是什么?"老赵问。

"琥珀。"陆沉轻轻打开盒子。

里面躺着那块透明的晶体。晶体中央,封存着那一小片组织样本——那对在大分裂之夜冻死的情侣留下的痕迹。

"三百年前,我从他们身上取下了这块样本。"陆沉说,"我不知道为什么要保存它。也许只是一种本能——想要留住一些'纯粹人类'的东西。"

他小心翼翼地把盒子重新埋进方尖碑的基座。

"无论我们变成什么,都要记住我们曾经是什么。"陆沉低声说,"有一天,也许我们会需要它。"

老赵默默地看着这一切。

"我该回去了。"他转过身,"地下城的人还在等消息。"

"老赵。"陆沉的声音在他脑海中响起——不是通过空气,而是直接通过磁场。

"嗯?"

"告诉他们,碳基时代结束了。下一代必须学会拥抱铁与磁。必须长出……\textbf{磁骨}。"

他顿了顿。

"还有,告诉他们,小心真空。"

"什么意思?"

"我不知道。"陆沉闭上眼睛,"但林渡在飞升前说过一句话,我一直没弄明白。她说:'真空不是空的。它只是在睡觉。'"

老赵愣了一下,然后转身离开。他的脚步在碎石上发出清脆的声响,渐渐远去。

在他身后,两尊金属雕像在星光下闪烁着幽蓝的光芒。

地核深处,二十亿个曾经是人的能量,正在缓缓旋转。他们的"手脚"——银枝和她的同伴们铸成的导电层——嵌在地幔与地核的交界处,永远地连接着天与地。

新的心跳,开始了。

而在方尖碑的基座下,那块琥珀静静地等待着。

它不知道自己等待的是什么。

但有一天,它会知道的。

\vspace{2em}

\begin{techbox}{档案:Homo Ferrum(铁血人)}
    \textbf{样本编号}:001\\[0.5em]
    \textbf{原名}:陆沉\\[0.5em]
    \textbf{年龄}:338岁(磁场纪元315年)\\[0.5em]
    \textbf{特征}:碳基-金属基混合结构;神经系统与地磁场共振;意识可在有机体与超导回路间转移\\[0.5em]
    \textbf{状态}:存活。永久驻守于方尖碑基座。\\[0.5em]
    \textbf{附注}:基座下方封存有旧人类组织样本,代号"琥珀",来源为大分裂之夜冻死的一对情侣。用途待定。\\[0.5em]
    \textbf{备忘}:持续监测宇宙背景辐射。注意任何异常波动。林渡警告——"真空不是空的"——含义未知,需持续观察。
\end{techbox}

\vfill
\begin{center}
    \rule{0.3\textwidth}{0.5pt}\\[1em]
    {\large 第二部 · 完}
\end{center}

\end{document}