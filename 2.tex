\documentclass[a5paper, 11pt, openany]{ctexbook}

% ==========================================
% 1. 宏包加载与页面配置
% ==========================================
\usepackage[a5paper, hmargin=2cm, vmargin=2.2cm]{geometry}
\usepackage{fancyhdr}
\usepackage{titlesec}
\usepackage{xcolor}
\usepackage{tcolorbox}
\usepackage{setspace}
\usepackage{enumitem}
\usepackage{fontspec}
\usepackage{amsmath}
\usepackage{amssymb}
\usepackage{graphicx}
\usepackage{hyperref}

% hyperref设置
\hypersetup{
    colorlinks=true,
    linkcolor=black,
    urlcolor=blue,
    pdfauthor={作者},
    pdftitle={幻肢 - 磁场纪元系列II},
}

% ==========================================
% 2. 样式定制
% ==========================================

% --- 全局行距 ---
\onehalfspacing

% --- 颜色定义 ---
\definecolor{rust}{RGB}{183, 65, 14}
\definecolor{void}{RGB}{15, 15, 20}
\definecolor{cherenkov}{RGB}{0, 191, 255}
\definecolor{flesh}{RGB}{204, 119, 119}
\definecolor{scenecolor}{RGB}{120, 60, 40}

% --- 章节标题样式 ---
\ctexset{
    part = {
        format = \huge\bfseries\centering\color{void},
        nameformat = \huge,
        number = \chinese{part},
        aftername = \par\vspace{1em},
        beforeskip = 0pt,
        afterskip = 30pt,
    },
    chapter = {
        format = \Large\bfseries\centering\color{void},
        number = \chinese{chapter},
        name = {第,章},
        beforeskip = 10pt,
        afterskip = 25pt,
        titleformat = \sffamily,
    }
}

% --- 页眉页脚 ---
\pagestyle{fancy}
\fancyhf{}
\fancyhead[CE]{\small\kaishu 幻肢 \quad Phantom Limb} 
\fancyhead[CO]{\small\sffamily\leftmark}
\fancyfoot[C]{\small\thepage}
\renewcommand{\headrulewidth}{0.4pt}

% --- 自定义命令:场景标记 ---
\newcommand{\scene}[2]{%
    \par\vspace{2.5em}%
    \noindent{\sffamily\bfseries\small\color{scenecolor} ▓ 坐标:#1}\\%
    \noindent{\sffamily\bfseries\small\color{scenecolor} ▓ 时间:#2}%
    \par\vspace{1em}%
    \noindent%
}

% --- 档案/技术文档框 ---
\newtcolorbox{techbox}[1]{
    colback=black!5, 
    colframe=void, 
    title=#1,
    fonttitle=\bfseries\sffamily,
    sharp corners,
    boxrule=1pt,
    left=5pt, right=5pt, top=5pt, bottom=5pt
}

% --- 飞升者对话框 ---
\newtcolorbox{ghostbox}{
    colback=cherenkov!5, 
    colframe=cherenkov, 
    boxrule=0.5pt,
    arc=0pt,
    leftrule=3pt,
    rightrule=0pt,
    toprule=0pt,
    bottomrule=0pt,
    fontupper=\kaishu\color{void}
}

% ==========================================
% 3. 正文内容
% ==========================================
\begin{document}

% --- 封面 ---
\begin{titlepage}
    \thispagestyle{empty}
    \vspace*{3cm}
    \begin{center}
        {\Huge\bfseries 幻\quad 肢}\\[0.8cm]
        {\Large\textit{Phantom Limb}}\\[3cm]
        
        \rule{0.5\textwidth}{1.5pt}\\[2.5cm]
        
        {\large\kaishu 磁场纪元系列 · 第二部}\\[0.3cm]
        {\small The Magnetic Era Trilogy: Book II}\\[5cm]
    \end{center}
\end{titlepage}

% --- 版权页 ---
\newpage
\thispagestyle{empty}
\vspace*{\fill}
\begin{center}
    \small
    磁场纪元三部曲\\[1em]
    第二部:幻肢\\[2em]
    初版\\[4em]
\end{center}
\vspace*{\fill}
\newpage

% --- 题词页 ---
\thispagestyle{empty}
\vspace*{4cm}
\begin{center}
    \begin{minipage}{0.85\textwidth}
        \kaishu\large
        \noindent 当潜水员从深海极速上浮,血液中的氮气会因为压强骤降而沸腾,这叫潜水病。\\[1.5em]
        
        当意识从光速维度的"永恒",极速坠落回物质世界的"此刻",\\[0.5em]
        时间本身会变成毒素。\\[1.5em]
        
        他们并非那是所谓的"神",\\[0.5em]
        他们只是一群患了"时间潜水病"的幽灵。
        
        \vspace{2cm}
        \hfill —— 地下城首席机械师 陆沉
    \end{minipage}
\end{center}
\newpage

% --- 目录 ---
\tableofcontents
\newpage

% ==========================================
% 序章
% ==========================================
\chapter*{序章:机器的痉挛}
\addcontentsline{toc}{chapter}{序章:机器的痉挛}
\markboth{序章:机器的痉挛}{序章:机器的痉挛}

\scene{地下4200米,第9号地热自动采掘区}{磁场纪元315年}

最初的异常,是从机器学会"喊疼"开始的。

那是一台编号为T-900的重型盾构机,全长120米,重达两千吨。三百年来,它像一只不知疲倦的铁蚯蚓,在高温高压的地幔边缘啃食岩层,为地下城输送地热能。它的逻辑电路简单而坚固:挖掘、冷却、传输、循环。

但今天,它停了。

维修工老赵爬进控制舱时,嗅到了一股奇异的味道。不是机油味,也不是烧焦的绝缘皮味,而是一种……臭氧的味道。就像是暴雨过后的天空被塞进了这个狭窄的驾驶舱。

"系统自检。"老赵敲击着布满油污的键盘。

屏幕闪烁了一下,跳出的不是故障代码,而是一行疯狂抖动的波形图。频率极高,甚至超出了硬件的刷新率,在屏幕上留下了残影。

AI的声音不再是那个死板的合成女声,它变得尖锐、扭曲,混杂着巨大的电流噪声,仿佛有一万个人在同时通过这个扬声器说话:

\begin{ghostbox}
    "腿……我的腿……"\\
    "为什么没有感觉……好轻……真空好轻……"\\
    "给我重量!给我摩擦力!"
\end{ghostbox}

巨大的盾构机突然剧烈震动起来。这不是机械故障引发的震动,而是像生物痉挛一样的抽搐。所有的液压臂疯狂地向四周的岩壁抓去,不是为了挖掘,而是为了——拥抱。

它在试图抱紧岩石。它在试图把自己挤进岩缝里,越紧越好。

"紧急切断!"老赵吼道,伸手拉下了物理闸刀。

电源切断了。但机器没有停。

那台两千吨的钢铁巨兽在没有任何能源输入的情况下,依靠着残留在电路里的某种高能电荷,继续发出凄厉的金属尖啸。它的钻头疯狂旋转,摩擦产生的高温将岩石烧得通红。

在红光的映照下,老赵看到了控制屏上的热成像。那不是机器的结构图。那是一个蜷缩的人形。它寄生在主控芯片里,正在绝望地尖叫。

"它不是坏了。"身后传来了陆沉的声音。

这位年轻的特级机械师不知何时站在了舱门口,他的眼神比地底的岩石还要冷。

"它是'感染'了。"陆沉盯着那个疯狂的机器,"天上的那些东西,掉下来了。"

% ==========================================
% 第一部
% ==========================================
\part[真空的潜水病]{真空的潜水病\\{\normalsize\textit{Vacuum Decompression Sickness}}}

\chapter{溃烂的天顶}

\scene{地下城"新长安",监听塔}{磁场纪元315年}

陆沉摘下耳机,耳膜还在隐隐作痛。

作为地下城唯一的"守墓人"(这是人们对地表观测员的戏称),他负责监控那个已经死去的世界。但最近,死寂被打破了。

"信息辐射(Information Radiation)。"陆沉把一张频谱图推到老赵面前,"地表现在的辐射值是正常值的五万倍。但不是伽马射线,不是X射线,是有序的熵流。"

老赵眯着浑浊的眼睛看着那张图:"你是说,那些'飞升者'在向我们广播?"

"不,他们在尖叫。"陆沉指着那些密集的波峰,"三百年前,他们抛弃肉体,化身为光,去往猎户座寻找新家园。我们以为他们成神了。但现在看来,他们失败了。"

"失败?"

"宇宙太大了,也太干净了。"陆沉的声音低沉,"纯能量体没有边界。在这个真空的宇宙里,他们的意识像一滴墨水滴进大海,不断稀释、扩散。他们失去了'我'的轮廓。"

陆沉打开了地表监视器的遮光板。屏幕上显示出地表的景象。

那是一幅地狱般的画面。天空不再是黑色的,而是呈现出一种病态的霓虹色。无数团发光的气旋在云层中翻滚,它们像是有生命一样,疯狂地撞击着地表。

每一次撞击,都没有物理爆炸,但地面上的岩石会被瞬间晶体化。那是被过量信息"写入"的结果。

"他们在寻找锚点。"陆沉说,"他们在太空中患上了'时间潜水病'。光速世界里没有时间,瞬间即永恒。这种无限的永恒把他们逼疯了。现在他们想回到低速世界,想回到有重力、有时间、有死亡的地方。"

"就像潜水员必须进减压舱?"老赵问。

"对。但他们下潜得太快了。"陆沉看着屏幕上那些疯狂的光团,"他们把地球当成了减压舱。但他们携带的信息量太大了。对于地球来说,这不是回归,这是\textbf{格式化}。"

\chapter{重力的止痛剂}

警报声在那个深夜拉响。

这一次不是单一的盾构机。整个地下城的芯片都在融化。

自动门疯狂开合,仿佛在试图呼吸。空气循环系统泵入了过量的氧气,导致数十人氧中毒。全息广告牌上的女郎停止了微笑,眼球开始疯狂转动,用一种古老的摩斯电码眨着眼睛:

\textbf{重。给我重。给我痛。}

陆沉穿上他的外骨骼装甲"塔耳塔洛斯",冲向核心服务器室。但他发现,不需要他动手了。

那些试图入侵的"光",正在遭遇惨烈的\textbf{排异反应}。

一名感染了"数据病毒"的技工正在地上打滚。他被一个飞升者的意识附体了。

"啊啊啊啊——"技工发出凄厉的惨叫。

"他在攻击人?"老赵惊恐地问。

"不。"陆沉看着读数,"是那个飞升者在惨叫。"

那个入侵的意识,在接触到人类大脑的一瞬间,被一种久违的东西击溃了——\textbf{局限性}。神经信号的延迟、关节的摩擦、重力的压迫、内脏的蠕动……这些对于人类来说习以为常的生理噪音,对于适应了量子纠缠的飞升者来说,简直就是酷刑。

就像一个习惯了飞翔的人,突然被塞进了一个灌满水泥的棺材里。

"好窄!好挤!好痛!"技工口中吐出不属于他的声音。紧接着,他的七窍流出了银色的液体——那是脑组织在高能数据流冲刷下溶解的产物。

飞升者的意识崩溃了,逃离了那具躯壳。留下一具脑死亡的尸体。

陆沉看着这一幕,握紧了拳头。"他们想回来做人,但他们已经忘了怎么做人。"

% ==========================================
% 第二部
% ==========================================
\part[免疫风暴]{免疫风暴\\{\normalsize\textit{Cytokine Storm}}}

\chapter{虚空的瘙痒}

为了阻止这场灾难,陆沉决定去地表谈判。既然"神"无法适应凡人的躯壳,他就要去天上问问他们到底想要什么。

他启用了"盲视"系统——一种不直接通过光学镜头,而是通过引力波侧影来观察目标的雷达。因为直视那些高能意识体,会直接烧毁视网膜和视神经。

在地表那座黑色的方尖碑前(第一部留下的遗迹),陆沉停下了脚步。方尖碑周围,光辉如雨。

陆沉打开了广域广播,用只有纯能量体能听懂的磁场波段喊话:"我是陆沉。我代表'石之民'。停止入侵,否则我们将引爆所有热核矿井,炸毁地球这个唯一的锚点。"

光雨停滞了片刻。随后,一个声音直接在陆沉的颅骨内共振。

\begin{ghostbox}
    \textit{"陆沉……我们……好冷……"}
\end{ghostbox}

是林渡。那个三百年前带领人类飞升的女人。现在的她,只是一团破碎的算法集合。

"你们自由了,为什么还要回来?"陆沉质问。

\begin{ghostbox}
    \textit{"自由是毒药。全知是诅咒。"}\\
    \textit{"我们看到了宇宙的尽头。那里什么都没有。只有死寂的热寂。"}\\
    \textit{"我们需要墙。我们需要一个笼子。我们需要感觉到'此时此刻'。"}
\end{ghostbox}

陆沉沉默了。他突然理解了这种悲哀。人类拼命想要打破枷锁,可当枷锁真的消失时,才发现那是唯一的安全带。

"我不能让你们进入地下城。"陆沉说,"你们的体量太大,会撑爆我们的文明。"

\begin{ghostbox}
    \textit{"那给我们造一个墓。或者一个监狱。只要是封闭的,只要是有限的。求求你……"}
\end{ghostbox}

\chapter{时间的高压氧舱}

回到地下城,陆沉提出了一个疯狂的计划。

"你要把方尖碑改成什么?"老赵以为自己听错了。

"\textbf{事件视界模拟器}(Event Horizon Simulator)。"陆沉在图纸上画出一个复杂的磁场闭环,"方尖碑是强相互作用力材料,它是宇宙中最完美的墙。我要在它内部制造一个高强度的引力透镜场。"

"用来做什么?"

"用来\textbf{降速}。"陆沉的眼神狂热,"我在里面把光速降低到每秒30万公里以下,制造巨大的人工重力,模拟出物质世界的'粘稠感'。对于那些患了'时间潜水病'的飞升者来说,那里就是最好的减压舱。"

"这需要巨大的能量来维持磁场约束。"老赵指出致命缺陷,"我们的地热井功率不够。"

"我有能源。"陆沉指了指自己。准确地说,是指向他外骨骼背部那个未经安全认证的实验性反应堆接口。

"你是想……"

"我会把自己变成\textbf{灯芯}。"陆沉平静地说,"我将把我的神经系统直接接入地核的残余磁场,用我的生物电来引导能量汇聚。只有人脑的复杂性,才能编织出能困住那些'神'的笼子。"

"你会死的。"

"不,我会变成别的东西。"

% ==========================================
% 第三部
% ==========================================
\part[金属骨痂]{金属骨痂\\{\normalsize\textit{The Metal Callus}}}

\chapter{神经嫁接}

\scene{地表,方尖碑基座}{磁场纪元315年}

陆沉最后一次看了一眼天空。那片溃烂的天空,充满了令人作呕的绚丽。

他将粗大的神经导管刺入了自己的脊椎。并没有想象中的剧痛,只有一种冰冷的麻木。那是人类神经被更高维度的能量接管的感觉。

"启动。"他下令。

方尖碑亮了。不是反光,而是内部激发出的幽蓝力场。强相互作用力材料表面开始震荡,发出低沉的嗡鸣,那是引力波在歌唱。

陆沉的意识瞬间被拉长。他感觉自己变成了那座碑。他感觉到了内部空间的弯曲、折叠。他正在用自己的意志,搭建一个虚拟的宇宙。

"进来吧!"陆沉对着天空怒吼。

漫天的光雨像是听到了召唤的飞蛾,疯狂地向方尖碑涌来。它们钻入碑体,钻入那个陆沉为它们精心打造的"低维天堂"。

每吸收一个光团,陆沉的身体就发生一次剧变的震颤。高能粒子流穿过他的身体,并没有烧毁他,而是发生了奇异的\textbf{冷聚变}。

他的血液停止流动,红血球被磁性纳米颗粒取代。他的骨骼在巨大的磁压下重组,钙质变成了金属晶格。他的皮肤硬化,变成了吸收辐射的半导体涂层。

他正在失去作为人的柔软。他正在获得作为物的永恒。

\chapter{超导体的守灵}

风暴停息了。

天空变得干净而黑暗。那些乱码般的极光消失了,取而代之的是久违的、冰冷的星空。

方尖碑矗立在荒原上,表面流动着一层仿佛水银般的银色光辉。它不再是一块死石头,它变成了一个巨大的、跳动的脏器。

而在方尖碑的基座上,坐着一个人。或者说,一尊雕像。

救援队赶到时,老赵试图唤醒陆沉。"陆沉?结束了吗?"

那尊"雕像"缓缓转过头。动作僵硬,伴随着金属摩擦的轻响。他的双眼已经没有了瞳孔,只剩下两点幽蓝的磁弧,深不见底。

"结束了。"陆沉的声音不再通过空气震动传播,而是通过磁场直接在老赵的脑海中响起。那种声音冷硬、清晰,没有任何感情色彩。

"他们在哪里?"老赵问。

陆沉抬起那只已经完全金属化的手,指了指身后的方尖碑。

"在里面。二十亿个做梦的鬼魂。我给了他们一个虚拟的旧地球。在那里,有重力,有生老病死,有爱恨情仇。他们很满意。"

"那你呢?"老赵看着陆沉那半人半机器的脸,泪水流了下来,"你还要在这里坐多久?"

陆沉没有回答。他现在的逻辑电路里,已经没有"多久"这个概念了。

他只是调整了一下坐姿,让自己与方尖碑的磁场连接得更紧密一些。源源不断的热能——那是飞升者在虚拟世界中活动产生的"废热"——正通过他的身体,转化为电能,输送回地下的新长安城。

他成了连接两个世界的半导体。他是光与石的中间态。他是唯一的\textbf{典狱长},也是唯一的\textbf{灯芯}。

% ==========================================
% 终章
% ==========================================
\chapter*{终章:第一块铁骨}
\addcontentsline{toc}{chapter}{终章:第一块铁骨}
\markboth{终章:第一块铁骨}{终章:第一块铁骨}

\scene{地下城实验室}{磁场纪元315年,尾声}

危机解除了。但人类的命运已经被永久改变。

老赵坐在显微镜前,观察着从陆沉身上取下的细胞样本。那不再是脆弱的有机细胞。细胞壁变成了高强度的碳纤维结构,细胞核内充满了能够存储海量数据的磁性颗粒。它不需要氧气,直接吞噬辐射能为生。

"这不是病变。"老赵的手在颤抖,声音却带着一种对未知的敬畏。

旁边的年轻助手问:"那是什么,老师?"

老赵抬起头,看向头顶厚重的岩层。他知道,在岩层之上,那个半金属的男人正独自坐在永恒的长夜里。

"这是\textbf{蓝图}。"老赵轻声说。

"碳基的时代结束了。为了在这个残酷的宇宙活下去,我们的下一代,必须长出铁的骨头。"

\vspace{3cm}

\begin{techbox}{进化档案:Homo Semiconductor(半导体人)}
    \textbf{样本编号}:001(陆沉)\\[0.5em]
    \textbf{特征}:
    \begin{itemize}[leftmargin=*, itemsep=2pt]
        \item 碳基-金属基混合结构
        \item 神经系统量子化,痛觉丧失,情感模块离线
        \item 能够直接感知并操控电磁场
        \item 能量来源:直接辐射吸收 / 磁共振
    \end{itemize}
    \textbf{状态}:存活。作为地表唯一的能量枢纽,监视着方尖碑内的文明模拟进程。
\end{techbox}

\vfill
\begin{center}
    \rule{0.3\textwidth}{0.5pt}\\[1em]
    {\large 第二部 · 完}
\end{center}

\end{document}