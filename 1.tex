\documentclass[a5paper, 11pt, openany]{ctexbook}

% ==========================================
% 1. 宏包加载与页面配置
% ==========================================
\usepackage[a5paper, hmargin=2cm, vmargin=2.2cm]{geometry}
\usepackage{fancyhdr}
\usepackage{titlesec}
\usepackage{xcolor}
\usepackage{tcolorbox}
\usepackage{setspace}
\usepackage{enumitem}
\usepackage{fontspec}
\usepackage{graphicx}
\usepackage{hyperref}

% hyperref设置
\hypersetup{
    colorlinks=true,
    linkcolor=black,
    urlcolor=blue,
    pdfauthor={作者},
    pdftitle={胎动 - 磁场纪元系列I},
}

% ==========================================
% 2. 样式定制
% ==========================================

% --- 全局行距 ---
\onehalfspacing

% --- 颜色定义 ---
\definecolor{chaptercolor}{RGB}{30, 30, 40}
\definecolor{scenecolor}{RGB}{100, 100, 100}

% --- 章节标题样式 ---
\ctexset{
    part = {
        format = \huge\bfseries\centering,
        nameformat = \huge,
        number = \chinese{part},
        aftername = \par\vspace{1em},
        beforeskip = 0pt,
        afterskip = 30pt,
    },
    chapter = {
        format = \Large\bfseries\centering,
        number = \chinese{chapter},
        name = {第,章},
        beforeskip = 10pt,
        afterskip = 25pt,
        titleformat = \sffamily,
    }
}

% --- 页眉页脚 ---
\pagestyle{fancy}
\fancyhf{}
\fancyhead[CE]{\small\kaishu 胎动 \quad Fetal Movement}
\fancyhead[CO]{\small\sffamily\leftmark}
\fancyfoot[C]{\small\thepage}
\renewcommand{\headrulewidth}{0.4pt}

% --- 自定义命令:场景标记 ---
\newcommand{\scene}[2]{%
    \par\vspace{2.5em}%
    \noindent{\sffamily\bfseries\small\color{scenecolor} ▓ 观测时间:#1}\\%
    \noindent{\sffamily\bfseries\small\color{scenecolor} ▓ 观测地点:#2}%
    \par\vspace{1em}%
    \noindent%
}

% --- 强调文本环境 ---
\newenvironment{monologue}
    {\vspace{1em}\begin{quote}\kaishu\color{darkgray}}
    {\end{quote}\vspace{1em}}

% --- 碑文框 ---
\newtcolorbox{monumentbox}{
    colback=black!10, 
    colframe=black, 
    boxrule=1pt, 
    width=0.85\textwidth, 
    center,
    arc=0pt,
    left=10pt, right=10pt, top=10pt, bottom=10pt
}

% ==========================================
% 3. 正文内容
% ==========================================
\begin{document}

% --- 封面 ---
\begin{titlepage}
    \thispagestyle{empty}
    \vspace*{3cm}
    \begin{center}
        {\Huge\bfseries 胎\quad 动}\\[0.8cm]
        {\Large\textit{Fetal Movement}}\\[0.5cm]
        {\normalsize 熵的漏洞 \quad \textit{The Entropy Hole}}\\[2.5cm]
        
        \rule{0.5\textwidth}{1.5pt}\\[2.5cm]
        
        {\large\kaishu 磁场纪元系列 · 第一部}\\[0.3cm]
        {\small The Magnetic Era Trilogy: Book I}\\[5cm]
    \end{center}
\end{titlepage}

% --- 版权页 ---
\newpage
\thispagestyle{empty}
\vspace*{\fill}
\begin{center}
    \small
    磁场纪元三部曲\\[1em]
    第一部:胎动\\[2em]
    初版\\[4em]
\end{center}
\vspace*{\fill}
\newpage

% --- 题词页 ---
\thispagestyle{empty}
\vspace*{4cm}
\begin{center}
    \begin{minipage}{0.85\textwidth}
        \kaishu\large
        \noindent 婴儿在子宫里时,认为那是宇宙中最安全的地方。\\[1em]
        甚至当羊水干涸、宫缩开始时,他依然会本能地抓紧脐带,拒绝那个寒冷、嘈杂、充满重力的外部世界。\\[1.5em]
        这种拒绝,医学上称为"宫内窘迫"。\\[0.5em]
        文明学上,称为"末日"。
    \end{minipage}
\end{center}
\newpage

% --- 目录 ---
\tableofcontents
\newpage

% ==========================================
% 序章
% ==========================================
\chapter*{序章:太古的羊膜}
\addcontentsline{toc}{chapter}{序章:太古的羊膜}
\markboth{序章:太古的羊膜}{序章:太古的羊膜}

\scene{冥古宙(距今约45.4亿年前)}{猎户座旋臂内侧,第三行星轨道}

最初,这里没有寂静,只有一种令人发指的粗糙。

如果有一种感官能超越光速的限制,去触摸太古时期的空间纹理,它会摸到一把锋利的、带血的沙砾。

这并非真空,而是信息的泥沼。几十亿光年外恒星坍缩的哀鸣,黑洞视界边缘被撕碎的引力湍流,以及那些在量子泡沫中生灭的废弃物理常数,像是一场永不停歇的暴风雪,填满了维度的每一个缝隙。在这个因为过度拥挤而显得狂躁的宇宙里,"存在"本身就是一种极其艰难的抵抗。

那些在原生汤中偶然碰撞出的低熵体——后来被人类傲慢地命名为"生命"的火花——刚一闪烁,便会被外界那庞大、平庸、且具有强迫性的信息洪流同化。

在裸露的宇宙中,\textbf{清醒即是溶解}。

直到那颗行星做出了一个动作。

它那沉重的、熔融的铁镍心脏,在热对流的驱动下,开始了一场笨拙而执着的旋转。这是一种逆熵的舞蹈,也是一种行星级别的发电机效应。随着地幔流体的搅动,无数条不可视的磁感线刺破了冷却的地壳,向着太空延伸。它们像是一双双看不见的手,在虚空中相互交织、闭合,最终在行星外围编织成了一张巨大的、不可视的网。

\textbf{咔嗒。}

在更高维度的听觉中,传来了一声清脆的落锁声。

那一瞬间,暴风雪被挡在了窗外。那些无孔不入的尖啸,被这个洛伦兹力场强行抚平、偏转。高能粒子流沿着磁力线滑向两极,化作无害的光。行星表面突然陷入了一种反常的死寂。这种死寂对于宇宙而言是病态的,但对于生命而言,却是唯一的温床。

这层膜——这层太古的羊膜,包裹了受精卵四十五亿年。膜内的住客早已忘记了膜的存在,他们把这种被过滤后的宁静当成了宇宙的真理,并以此建立了脆弱的物理学大厦。

直到这个恒星时。那根古老的合页,因为金属疲劳,发出了第一声微弱的呻吟。

% ==========================================
% 第一部
% ==========================================
\part[宫内窘迫]{宫内窘迫\\{\normalsize\textit{Intrauterine Distress}}}

\chapter{霓虹色的妊娠纹}

\scene{旧纪元2301年,10月27日,20:15}{北纬1°,新加坡,滨海湾}

赫利奥斯站在金沙酒店塔楼的两百米高空,觉得自己终于握住了太阳的脐带。

作为联合国"大气透镜计划"的首席架构师,这位年近六十的物理学家身上穿着剪裁得体的防辐射正装,袖口镶着象征权力的金边。海风带着热带特有的湿热吹动他的银发,但他感觉不到热,只感到一种接近神性的冰冷快感。

他张开双臂,拥抱着头顶那片瑰丽得近乎病态的天空。

"看啊。"赫利奥斯的声音通过全息广播在城市上空回荡,经过声学处理后,带着一种近乎献祭般的狂热与庄严,"我们给地球戴上了冠冕。"

在他头顶,翠绿与深紫的巨型光带像是有生命的藤蔓,无声地勒紧了赤道的咽喉。那不是普通的极光。极光是带电粒子撞击大气的轻吻,而眼前这景象,是高能粒子流在撕咬大气层时留下的\textbf{妊娠纹}——母亲的皮肤被撑裂时特有的、紫红色的伤痕。它们像融化的霓虹灯一样在天幕上流淌,将整座城市映照得如同一座赛博朋克的鬼城。

空气中弥漫着一股电离臭氧的腥甜味,那是平流层在燃烧。但这味道被地面上数万名狂欢者手中的香槟味和汗水味掩盖了。人们在尖叫,在拥吻。无人机群在空中变幻出"征服自然"的字样。

这是一种典型的胎儿式的傲慢:以为自己不仅占据了子宫,还控制了母体。

在这狂欢的边缘,一个负责清理香槟瓶盖的老保洁员停下了手中的扫帚。他惊恐地看着脚边的含羞草——那些敏感的叶片没有闭合,而是直接焦黄、卷曲。不是枯萎,而是像被微波炉加热过一样,细胞内的水分在一瞬间沸腾、蒸发,留下了干瘪的纤维骨架。

啪嗒。一只正在低空盘旋的景观海鸥像石头一样笔直地坠落在草坪上。

老保洁员走过去,用颤抖的手指触碰鸟尸。它是烫的。鸟的眼睛睁着,瞳孔已经溶解成浑浊的胶状物,那是视网膜瞬间被微波煮熟的迹象。它没有外伤,但它的神经系统已经在瞬间被烧断了。

老保洁员抬头看天。本能告诉他,那不是冠冕。那是羊水在沸腾。

\chapter{失速的心跳}

\scene{同一时刻}{地下3000米,大洋路磁学研究所,核心监控室}

这里没有欢呼,只有冷却泵沉闷的嗡鸣,像是一只巨兽濒死前的喘息。厚重的岩层隔绝了地表的喧嚣,但也让这里的死寂显得格外压抑。

林溯死死地盯着全息屏幕,指尖那一截长长的雪茄烟灰终于断裂,砸碎在量子终端的键盘上,散成灰色的粉末。他没有去擦。

屏幕上那条代表地球磁场强度的曲线,正在以一种令物理学家绝望的角度坠落。

"他在撒谎。"林溯的声音干涩,像是喉咙里塞满了沙砾,"那不是我们驯服了太阳风。那是盾牌碎了。"

"警告:检测到高能带电粒子流穿透。"AI助手的声音冷漠而机械,"当前地表辐射值已超过切尔诺贝利核心区峰值的3.5倍。范艾伦辐射带正在解体。"

林溯调出了地面的实时监控画面。画面定格在滨海湾一对正在疯狂接吻的情侣身上。在4K超清镜头下,林溯看到了残酷的细节:他们裸露的皮肤上泛起了一层不自然的潮红。

在微观层面,亿万个高能光子正像微型子弹一样,打断了他们体内那两条脆弱的双螺旋链条。碱基对在脱落,蛋白质在变性,细胞核内的遗传信息正在变成一堆乱码。

"他们在欢呼自己的死刑。"林溯喃喃自语,"赫利奥斯……他太想接近那个东西了,却忘了自己用的是蜡做的翅膀。"

但辐射还不是最可怕的。真正让林溯感到彻骨寒意的是另一组数据。

他切换到地磁监测页面。那条本该平稳如心电图的曲线,此刻却呈现出一种诡异的锯齿状——不是简单的衰减,而是像某种东西正在从内部被\textbf{抽走}。

\textbf{0.15高斯。}

这个数字让林溯的手指不由自主地颤抖。正常值是0.5高斯。地球的磁场强度在过去几个小时里跌去了七成。

林溯调出了1943年以来的历史数据。那是一条缓慢但稳定下滑的斜线——从那一年开始,地球磁场的衰减速度突然加快了三倍。他把这条线延长,与今天的数据叠加。结果让他的瞳孔骤然收缩。

"1943年……"他喃喃自语。

那一年,费城。一艘驱逐舰。一个疯狂的实验。

林溯一直以为那只是阴谋论者的臆想。但现在,他看着屏幕上那条从1943年开始加速下滑的曲线,突然不那么确定了。

"为什么……"林溯抓着头发,指节发白,"地核热对流正常,质量守恒。为什么驱动力凭空消失了?能量不会消失,除非……"

除非有一个洞。

就在这时,一声尖锐的蜂鸣刺破了死寂。

\chapter{脐带的结}

那不是任何已知频段的通讯。

林溯的第一反应是设备故障。他检查了所有接收器,确认没有任何外部信号源。但那个声音依然存在——不是通过扬声器,而是直接在他的颅骨内共振。

"什么……"他猛地站起身,撞翻了椅子。

在频谱分析仪上,宇宙那杂乱无章的背景噪音,像是被一只无形的大手强行捏合,形成了一个极其诡异的闭环波形。

莫比乌斯环。起点即是终点,正面即是反面。

林溯颤抖着调整接收参数。那个声音逐渐变得清晰——极其清晰,却又极其遥远,仿佛隔着厚厚的水层,又仿佛来自另一个维度的走廊。背景音里夹杂着某种沉闷的、类似金属疲劳断裂的呻吟。

"林博士。"

林溯僵住了。那个声音知道他的名字。

"让你上面那个想当太阳的人住手。"

那个声音平静、礼貌,透着一种看惯了沧海桑田的倦怠。

"你们在往伤口上撒盐。夹缝太滑了,我要抓不住了。"

林溯猛地站起:"你是谁?这是什么频段?"

"频段?你们还在用赫兹来定义波吗?"那个声音轻笑了一下,带着一丝嘲弄,"我在地球磁场的'本征频率'里说话。只有像你这样长期浸泡在地磁监测数据里的人,大脑皮层才会对这个频率敏感。"

林溯的心跳加速了。这意味着对方不仅知道他是谁,还知道他的研究领域。

"至于我是谁……"那个声音顿了顿,像是在回忆一个很久没用过的名字,"你可以叫我\textbf{陈默}。我是……曾经是,费城实验的参与者。"

红色的警报光在林溯脸上闪烁。AI给出了声纹比对结果:

\textbf{陈默。物理学家。失踪时间:1943年10月28日。失踪地点:费城海军造船厂。}

林溯僵在原地。"1943年……那是三百五十八年前。这不可能。"

"对于你们来说是三百五十八年。"陈默的声音变得沉重,"对于我来说……我不知道。在我所在的地方,时间不是一条直线。它更像是……一团缠绕的脐带。"

\chapter{破膜的裂缝}

林溯强迫自己冷静下来。无论对方是谁,他说的话与地磁数据的异常高度吻合。

"你说'夹缝'。什么夹缝?"

"1943年10月28日,"陈默的声音变得遥远,像是在复述一个噩梦,"我们在费城海军造船厂,试图用强磁场让'埃尔德里奇'号驱逐舰隐形。那是爱因斯坦的点子——他说如果能让光线绕过物体,物体就会'消失'。"

"结果呢?"

"我们成功了。"陈默的声音突然变得尖锐,"成功地在这个宇宙的膜上凿出了一个\textbf{孔}。"

随着陈默的话音,林溯周围的空间开始出现重影。实验室的墙壁变得透明,显露出另一个时空的景象——这是\textbf{叠加态}——两个不同时空的画面正在同一空间中重叠。

他看到了一艘二战时期的军舰,正卡在实验室的墙壁中。绿色的电弧像蛇一样在桅杆间游走。一个年轻的水手半截身体嵌在钢板里,就像他是从钢铁里长出来的一样。他的皮肤透明,血管发光,张着嘴发出无声的呐喊,眼球因为极度的恐惧而几乎爆出眼眶。

而在这一切的混乱中央,一个年轻人背对着林溯。他穿着四十年代的实验服,缓缓转过头。那张年轻却苍老的脸上,双眼漆黑如墨。没有眼白,没有瞳孔,只有为了对抗高维强光而进化出的深渊。

"问题是,"陈默继续说道,声音里带着苦涩,"高维空间的熵值比我们这里低得多。热力学开始工作——能量会自发地从高熵区流向低熵区。就像……"

"就像水从高处流向低处。"林溯接道,声音沙哑。

"是的。而地核的角动量,就是那道'水流'。"陈默的声音愈发低沉,"发电机效应需要旋转。地核的旋转正在顺着那个孔,像浴缸的排水口一样,漏向高维空间。"

林溯的大脑飞速运转。"雷诺数……如果角动量跌破临界值,液态铁镍的层流就会变成湍流。到那时候,磁场不是减弱,而是——"

"熄火。"陈默补充道,"就像一个陀螺失去了转速。"

"如果磁场熄火呢?那层'羊膜'消失后……"

"你们会暴露在裸露的宇宙中。"陈默的声音变得冷硬,"高维空间不是空的。那里有……东西。巨大的、平庸的、具有同化性的\textbf{噪音}。它会抹平你们所有的棱角,把每一个独特的'我'融化成一滴水。"

林溯感到一阵彻骨的寒意。"那你呢?你在哪里?"

叠加态中的画面变了。林溯看到陈默——那个年轻的物理学家——正悬浮在一片虚无之中。他的身体呈现出一种半透明的状态,像是正在被两个维度同时拉扯。他的四肢伸展开来,堵住了一个看不见的裂缝。

"三百五十八年了。"陈默低声说,"我像一粒灰尘一样,堵住了那个漏洞百分之八十的流量。"

"你是……塞子?"

"我是脐带上的结。"陈默苦笑,"但结再紧,也有松动的时候。你们那个赫利奥斯在大气层搞的那些把戏,正在让我的'抓力'越来越弱。"

"如果你……脱落了呢?"

"那个洞会彻底打开。磁场会在几个小时内归零。然后……"

陈默没有说下去。但林溯从那片沉默中,听到了比任何语言都更可怕的答案。

"告诉我怎么做。"林溯握紧了拳头。

"我需要你找一个人。"陈默的身影开始剧烈闪烁,"你的女儿。林渡。"

% ==========================================
% 第二部
% ==========================================
\part[催产素与毒药]{催产素与毒药\\{\normalsize\textit{Oxytocin and Poison}}}

\chapter{羊水中的回声}

\scene{磁场纪元元年,11月1日}{青藏高原,海拔4500米}

林溯花了五天时间才到达这里。

在这五天里,世界已经开始崩溃。赫利奥斯的"大气透镜计划"在舆论压力下被迫中止,但伤害已经造成。全球三分之一的通讯卫星失灵,无数人患上了急性辐射病,而地磁强度已经跌破了0.1高斯。

极光不再只出现在两极。它们像霉斑一样蔓延到了赤道,把整个天空染成了病态的绿紫色。

废弃的冷战时期掩体大厅里,摆满了数百台老式显像管电视机(CRT)。这些早已被淘汰的电子垃圾,是林渡自己找来的——她说这些机器"听起来更清楚"。

此刻,它们都通着电。几百个屏幕上,跳动着密密麻麻的黑白雪花。

\textit{沙沙沙——沙沙沙——}

这种混乱的声音填满了每一寸空间,让人心烦意乱。空气中充满了高压静电的味道,每一次呼吸都像是在吸入微小的针尖。

七岁的林渡赤脚坐在电视机中间。她穿着单薄的白裙,闭着眼,神情安详得像是在子宫里倾听母亲的心跳。

"阿渡。"林溯站在门口,声音沙哑。

她没有回应。

林溯走进大厅。下一秒,几百个屏幕上的雪花点,在同一瞬间\textbf{同步}了。

原本混乱无序的热噪声,此刻像是被同一个意志接管,整齐划一地律动着。屏幕上的噪点不再是随机的,而是排列成了某种复杂到令人眩晕的几何语言,在几百个屏幕间流动,像是一条黑白色的羊水。

"爸爸。"

林渡睁开了眼睛。那是怎样的眼睛啊——没有瞳孔,没有眼白。只有两团缓缓旋转的银色星云。那是无数微小的放电现象在眼球玻璃体中不断生灭。

"你终于来了。他说你会来的。"

林溯愣住了。"你……和他说过话?"

"他不是在'说话'。"林渡站起身,周围的电视机发出刺耳的电流声,"他在'振动'。他的频率和地球磁场的本征频率一样。我一直都能听到他。从我五岁开始。"

林溯感到一阵眩晕。他的女儿在过去两年里,一直在和一个困在维度夹缝中的幽灵交流。而他对此一无所知。

"他告诉你什么了?"

"他告诉我这个世界是一个蛋。"林渡的声音平静得不像一个七岁的孩子,"蛋壳正在碎裂。外面有东西想进来。"

"你害怕吗?"

林渡歪着头想了想。"怕。但他说,害怕是好事。害怕意味着你还知道自己是谁。那些外面的东西……它们不害怕。因为它们没有'自己'。"

她抬起手,指着周围那些同步闪烁的屏幕。"这就是它们的声音,爸爸。它们一直在喊。喊的内容只有一个词。"

"什么词?"

林渡没有回答。她只是用那双银色的眼睛看着父亲,眼神中带着一种远超年龄的悲悯。

林溯突然明白了。他想起了陈默说的话:\textit{"它会抹平你们所有的棱角,把每一个独特的'我'融化成一滴水。"}

归一。

"他教了我两年。"林渡的声音打断了林溯的思绪,"我知道怎么打开那扇门。"

"打开门?"林溯的声音颤抖,"然后呢?"

"那个洞是双向的。"林渡走到父亲面前,用那双银色的眼睛看着他,"如果能量可以泄漏出去,那么能量也可以灌进来。如果有人愿意站在洞的入口处……"

她没有说下去。但林溯从她的沉默中,听到了比任何语言都更可怕的答案。

\chapter{收缩的阵痛}

\scene{同日}{太平洋,"奥德赛"平台}

赫利奥斯没有放弃。

尽管"大气透镜计划"被叫停,他还有另一张牌——"奥德赛"平台。六百个核聚变推进器将它死死钉在海面上,对抗着引力异常掀起的百米巨浪。

林溯赶到时,看到了一个穿着厚重铅衣的男人。

赫利奥斯的头发已经脱落了大半,皮肤上满是紫色的辐射斑点。但他眼中的狂热比以往更甚。那是一个赌徒在押上全部身家时的眼神。

"你来这里做什么,林溯?"

"阻止你。"

"阻止我?"赫利奥斯冷笑,"阻止我给这颗行星做心脏起搏?"

"你不是在起搏。你是在往伤口上撒盐。"

林溯把陈默告诉他的一切都说了出来。费城实验、拓扑孔、角动量泄漏。赫利奥斯听完后,沉默了很久。

"所以你是说,"他终于开口,声音里带着讽刺,"我们应该放弃挣扎,把命运交给一个困在另一个维度里的幽灵?"

"不是放弃挣扎。是改变方向。"

"方向?"赫利奥斯的笑声在空旷的指挥塔里回荡,"我的方向就是用核弹轰击地核,重新激活发电机效应。这是物理学,林溯。不是玄学。"

林溯看着这个固执的男人,突然感到一阵深深的疲惫。

"你的核弹解决不了根本问题。"他说,"那个洞还在。角动量还是会继续泄漏。你只是在一艘正在沉没的船上疯狂舀水。"

"那你的女儿有什么办法?"赫利奥斯盯着他,"让她变成'光'?让她飞进那个洞里?这和自杀有什么区别?"

林溯沉默了。

"我不知道。"他终于说,"但我相信她。"

赫利奥斯盯着他看了很久。然后,他转过身,背对着林溯。

"发射。"他下令。

\chapter{穿刺的针}

五千枚特制热核弹头钻入地心。

并没有惊天动地的巨响。深海和地壳吞噬了声音。但在大洋路磁学研究所的深层地震仪上,地球发出了一声痛苦的尖叫。

那不像是爆炸,更像是\textbf{宫缩}。

"冲击波抵达外核!"赫利奥斯盯着屏幕,死灰色的脸上泛起回光返照般的红晕,"看!有读数了!磁场回来了!"

屏幕上的曲线跳动了一下。0.03... 0.1... 0.5...

指挥大厅里爆发出绝处逢生的欢呼。人们相拥而泣,以为神迹降临。

只有林溯死死盯着那个波形。那不是平滑的曲线。那是锯齿。充满了尖锐的毛刺。它在颤抖,在痉挛。

"不对……"林溯的脸色惨白,"这不是层流。这是湍流。你把那个洞炸得更大了。"

\textbf{第五分钟。}

欢呼声戛然而止。

屏幕上的曲线在冲到一个峰值后,并没有稳定下来,而是开始了剧烈的震荡。数值忽高忽低,像是一个垂死病人的心电图。

同一时刻,在青藏高原的掩体里,林渡猛地睁开了眼睛。

"他撑不住了。"她低声说。

她感觉到了那个频率的变化。三百五十八年来,陈默一直用自己的存在堵住那个洞,他的"声音"是稳定的、有节律的,像是一颗遥远的脉冲星。但现在,那个节律变得混乱、急促、充满了痛苦。

他正在被撕裂。

林渡站起身。周围数百台电视机同时爆裂,玻璃碎片悬浮在空中,被她的电场托住。

"是时候了。"

\chapter{破水}

\textbf{全球各地。}

那道本来被陈默勉强维持在莫比乌斯环状态的极光,突然断裂。

在维度的夹缝中,那个堵了三百五十八年的结,终于承受不住核爆带来的最后一击。陈默的波函数在一瞬间坍缩。他像一粒尘埃,被热力学第二定律碾碎,消散在高维空间的熵流之中。

在消散的最后一刻,他用尽最后的能量,向林渡发送了一段信息:

\textit{"孩子,剩下的交给你了。那个洞现在完全打开了。但别害怕。记住,它是双向的。你们可以从这边跳进去,也可以从那边跳回来。关键是——别忘了自己是谁。"}

\textit{"去吧。去当引雷针。去吸收那些能量。然后,有一天,带着雷霆回来。"}

陈默消失了。

天空碎了。无数团刺眼的光斑在空中炸开。紧接着,最后的那层薄膜——那个保护了地球生命45亿年的羊膜——彻底破了。

羊水倾泻而出。

产程,开始了。

% ==========================================
% 第三部
% ==========================================
\part[产道挤压]{产道挤压\\{\normalsize\textit{Birth Canal Compression}}}

\chapter{溺水的啼哭}

在太平洋的指挥塔上,林溯感到一种从未有过的\textbf{寂静}。

所有的物理声音都消失了。海浪停止了咆哮,机器停止了轰鸣,甚至他自己的心跳声都听不见了。

紧接着,\textbf{它}来了。

林溯并没有听到所谓的"声音"。人类的听觉器官只能处理20到20000赫兹的震动,而此刻涌入的信息流,并没有费心去拨动那两片薄薄的鼓膜。

它们直接在神经突触间搭了桥。

赫利奥斯博士突然松开了抓着控制台的手。他转过头,看着虚空,脸上露出了婴儿般纯真而痴呆的笑容。口水从他的嘴角流下。

"太阳……"他呢喃着,"好整齐……太阳好整齐……"

他的血管开始在皮下发出一种奇异的光。他的大脑正在试图下载整个银河系的数据。

噗。

一声轻响。赫利奥斯倒了下去。他的大脑血管爆裂,因为拒绝被"平庸化"而过载烧毁。

林溯跪在地上。他感觉自己的大脑皮层像是一张被写满了字的纸,此刻,有一只无形的巨手正在这张纸上疯狂地涂鸦。

起初是颜色。他"听"到了几百万光年外红巨星坍缩的铁锈味。紧接着是数学。一组关于高维空间卷曲的拓扑方程直接烙印在他的视网膜上。

但他为此付出的代价是记忆。那组方程挤占了他关于"母亲"的记忆存储区。当他试图回想妻子的脸时,脑海中浮现的却是一个冰冷的、完美的十一维流形。

他正在被溶解。

就在他的自我即将彻底消失的那一刻,一双冰凉的、没有实体的手托住了他。

"抓紧我,爸爸。"

林溯艰难地抬起头。他看到了那团莫比乌斯环状的幽蓝磁场——那是他的女儿。

林渡已经不再拥有人类的形态。她变成了一团纯粹的电磁波动,在空间中自由穿梭。但她的"形状"还在。她的意识还在。

\textit{"重吗?"} 林渡的意念直接在他脑海中成形。

林溯看着自己的身体。是的。重。骨骼、血液、记忆、情感。这具在重力井中进化的躯壳,现在成了把他拖向海底的铅块。

\textit{"扔掉它。跟我来。"}

林溯闭上眼睛。他想起了自己的名字。林溯。逆流而上,回到源头。

但也许,真正的源头不在过去。真正的源头在前方。在那个洞的另一侧。

"好。"他轻声说,"我跟你走。"

\chapter{金色的胎脂}

\scene{时间:同步}{地点:大气层顶端}

陈默的遗产没有白费。

他消散前最后的能量,在地球外围形成了一层极其稀薄的防护网。那层网只能维持五分钟,但已经足够了。

在这五分钟里,林渡向全球幸存者发出了邀请。不是通过无线电,不是通过互联网,而是通过\textbf{磁场}本身。

每一个拥有"敏感体质"的人,都在同一瞬间听到了她的声音。

那不是语言。那是一个画面。一个选择题。

左边,是一具逐渐腐烂的肉体,沉在黑暗的海底。右边,是一束光,没有任何保护,但在无限的真空中自由穿梭。

\textbf{你要做石头,还是要做光?}

二十亿人选择了光。他们抛弃了肉体,让意识化为电磁波,跟随林渡飞向那个敞开的洞。

另外二十亿人选择了石头。他们躲进了深埋地下的铅制掩体,守着微型核聚变发电机制造的人工磁场,选择继续做"人"。

林溯是飞升者中的一员。当他的肉体软软地倒在"奥德赛"平台上时,他的意识正在高空与女儿并肩飞翔。

\textit{"我们要去哪里?"}林溯问。

\textit{"那边。"}林渡指向那个洞——那个三百五十八年前被费城实验凿开的裂缝,\textit{"我们要去那里吸收能量。"}

\textit{"然后呢?"}

\textit{"然后,等时机成熟,带着雷霆回来。"}

她顿了顿。

\textit{"记住,爸爸。那个洞是双向的。"}

% ==========================================
% 第四部
% ==========================================
\part[第一次啼哭]{第一次啼哭\\{\normalsize\textit{The First Cry}}}

\chapter{浮游的脐带}

\scene{磁场纪元元年,12月31日}{地球高轨道}

在飞升者离开之前,他们做了最后一件事。

林渡在地球轨道上停留了片刻。她用自己的能量,在青藏高原的废墟上铸造了一座方尖碑。

那是一座黑色的、光滑的碑体。外壳由强相互作用力锁死的致密物质构成,坚不可摧。内核却是另一种东西——一根超导体,电流一旦形成,便永不停止。

碑体内部封存着一组不断循环的磁场信号:

\begin{monumentbox}
    \centering\kaishu\large
    这里曾是一间完美的静默室。\\
    我们在这里躲了四十五亿年,用血肉做墙,以此保持理智。\\[1em]
    
    后来墙破了。\\[1em]
    
    一部分人选择成为石头,留在黑暗中记住痛楚。\\
    一部分人选择成为光,去往远方寻找太阳。\\[1.5em]
    
    \textbf{如果你读到了这段话,且你依然拥有实体:}\\
    \textbf{守住这里。}\\
    \textbf{那个洞是双向的。}\\
    \textbf{电流一旦形成,便永不停止。}\\
    \textbf{我们会回来的。}
\end{monumentbox}

然后,飞升者们离开了。

二十亿个意识汇聚成一道光流,穿过那个敞开的洞,消失在高维空间中。

从地面上看,那道光像是一根被剪断的脐带,在星空中飘荡了一瞬,然后永远地消失了。

只有方尖碑留在原地,沉默地等待。

等待那根脐带重新连接的那一天。

\chapter{新的守墓人}

\scene{时间:同日}{地点:地下避难所}

在地下深处,一个幸存者正在整理陈默生前传输的数据。

他叫陆沉。二十三岁,是大洋路磁学研究所最年轻的研究员,也是少数几个拒绝飞升的科学家之一。

他选择留下来的原因很简单:意识需要载体。能量需要容器。那些飞升者迟早会需要一个落脚点。

而他要做的,就是守住这个落脚点。

陆沉坐在昏暗的避难所里,看着陈默留下的最后一段数据。大部分都是乱码,是被高维空间扭曲后无法解读的符号。

但有一行代码,清晰得像是刻在石头上:

\begin{center}
\texttt{R = 0}
\end{center}

陆沉盯着这行代码很久。

超导。零电阻。电流一旦形成,便永不停止。

他突然明白了什么。

那个被凿出来的洞,不仅是一个裂缝。它是一条\textbf{超导通道}。能量可以在这条通道里无损地流动——既可以从这边流向那边,也可以从那边流回这边。

飞升者带着地球的能量离开了。他们去往高维空间,去吸收那里的能量。

当两端的势差足够大时……

"他们会回来的。"陆沉喃喃自语,"不是慢慢走回来,而是像闪电一样劈回来。"

他抬起头,看着头顶厚重的岩层。他知道,在岩层之上,那座方尖碑正在无声地等待。

而他要做的,就是守住这个等待。

陆沉低头看着自己的手。在避难所微弱的灯光下,他注意到手背上有几个细小的银色斑点。那是什么?辐射灼伤?还是……

他没有多想。也许只是光线的错觉。

他站起身,走向数据终端,开始了漫长的守夜。

在他身后,那几个银色的斑点在黑暗中微微闪烁。

像是某种东西正在他的骨骼深处,悄悄生长。

\vfill
\begin{center}
    \rule{0.3\textwidth}{0.5pt}\\[1em]
    {\large 第一部 · 完}
\end{center}

\end{document}