\documentclass[a5paper, 11pt, openany]{ctexbook}

% ==========================================
% 1. 宏包加载与页面配置
% ==========================================
\usepackage[a5paper, hmargin=2cm, vmargin=2.2cm]{geometry}
\usepackage{fancyhdr}
\usepackage{titlesec}
\usepackage{xcolor}
\usepackage{tcolorbox}
\usepackage{setspace}
\usepackage{enumitem}
\usepackage{fontspec}
\usepackage{amsmath}
\usepackage{amssymb}
\usepackage{graphicx}
\usepackage{hyperref}

% hyperref设置
\hypersetup{
    colorlinks=true,
    linkcolor=black,
    urlcolor=blue,
    pdfauthor={作者},
    pdftitle={磁骨 - 磁场纪元系列III},
}

% ==========================================
% 2. 样式定制
% ==========================================

% --- 全局行距 ---
\onehalfspacing

% --- 颜色定义 ---
\definecolor{void}{RGB}{10, 10, 15}
\definecolor{bone}{RGB}{210, 215, 211}
\definecolor{core}{RGB}{255, 69, 0}
\definecolor{arc}{RGB}{100, 149, 237}
\definecolor{scenecolor}{RGB}{100, 100, 100}

% --- 章节标题样式 ---
\ctexset{
    part = {
        format = \huge\bfseries\centering\color{void},
        nameformat = \huge,
        number = \chinese{part},
        aftername = \par\vspace{1em},
        beforeskip = 0pt,
        afterskip = 30pt,
    },
    chapter = {
        format = \Large\bfseries\centering\color{void},
        number = \chinese{chapter},
        name = {第,章},
        beforeskip = 10pt,
        afterskip = 25pt,
        titleformat = \sffamily,
    }
}

% --- 页眉页脚 ---
\pagestyle{fancy}
\fancyhf{}
\fancyhead[CE]{\small\kaishu 磁骨 \quad Magnetic Bones} 
\fancyhead[CO]{\small\sffamily\leftmark}
\fancyfoot[C]{\small\thepage}
\renewcommand{\headrulewidth}{0.4pt}

% --- 自定义命令:场景标记 ---
\newcommand{\scene}[2]{%
    \par\vspace{2.5em}%
    \noindent{\sffamily\bfseries\small\color{scenecolor} ▓ 坐标:#1}\\%
    \noindent{\sffamily\bfseries\small\color{scenecolor} ▓ 时间:#2}%
    \par\vspace{1em}%
    \noindent%
}

% --- 磁骨人感应交流框 ---
\newtcolorbox{magnetbox}{
    colback=bone!10, 
    colframe=gray, 
    boxrule=0.5pt,
    arc=0pt,
    leftrule=2pt,
    rightrule=0pt,
    toprule=0pt,
    bottomrule=0pt,
    fontupper=\sffamily\small\color{void},
    title={\footnotesize $\lll$ Magnetic Resonance $\ggg$},
    coltitle=black,
    fonttitle=\bfseries
}

% --- 飞升者/方尖碑通讯框 ---
\newtcolorbox{voidbox}{
    colback=black!90, 
    colframe=black, 
    boxrule=0pt,
    arc=2pt,
    fontupper=\kaishu\color{cyan},
    halign=center
}

% --- 结尾献词框 ---
\newtcolorbox{finalbox}{
    colback=black, 
    colframe=white, 
    boxrule=1pt, 
    width=0.85\textwidth, 
    center,
    arc=0pt
}

% ==========================================
% 3. 正文内容
% ==========================================
\begin{document}

% --- 封面 ---
\begin{titlepage}
    \thispagestyle{empty}
    \vspace*{3cm}
    \begin{center}
        {\Huge\bfseries 磁\quad 骨}\\[0.8cm]
        {\Large\textit{Magnetic Bones}}\\[3cm]
        
        \rule{0.5\textwidth}{1.5pt}\\[2.5cm]
        
        {\large\kaishu 磁场纪元系列 · 第三部}\\[0.3cm]
        {\small The Magnetic Era Trilogy: Finale}\\[5cm]
    \end{center}
\end{titlepage}

% --- 版权页 ---
\newpage
\thispagestyle{empty}
\vspace*{\fill}
\begin{center}
    \small
    磁场纪元三部曲\\[1em]
    第三部:磁骨\\[2em]
    初版\\[4em]
\end{center}
\vspace*{\fill}
\newpage

% --- 题词页 ---
\thispagestyle{empty}
\vspace*{4cm}
\begin{center}
    \begin{minipage}{0.85\textwidth}
        \kaishu\large
        \noindent 雏鸟破壳时,唯一的动作就是毁坏。\\[0.5em]
        它必须吃掉蛋黄,必须啄碎蛋壳。\\[1.5em]
        
        对于宇宙中的文明而言,\\[0.5em]
        弑母,是生存的第一道工序。
        
        \vspace{2cm}
        \hfill —— 磁骨族领袖 铁心
    \end{minipage}
\end{center}
\newpage

% --- 目录 ---
\tableofcontents
\newpage

% ==========================================
% 序章
% ==========================================
\chapter*{序章:行星的尸僵}
\addcontentsline{toc}{chapter}{序章:行星的尸僵}
\markboth{序章:行星的尸僵}{序章:行星的尸僵}

\scene{地下8000米,末日避难所底层}{磁场纪元1024年}

地球死了。

这次是彻底的脑死亡。七百年前,陆沉将自己的神经接入地核,那是最后一次起搏。现在,那颗铁镍心脏终于停止了微弱的颤动,彻底冷却为一块巨大的磁铁。

没有了地热,地下城变成了冰窖。氧气合成机早在五十年前就停摆了。所有的碳基生物——那些还要呼吸氧气、喝水、吃蛋白质的旧人类,已经全部变成了冻土层里的化石。

在这个绝对零度的墓穴里,只有一种东西还在动。

铁心(Iron Heart)在黑暗中睁开了眼睛。他的眼球没有反光,因为那是两颗打磨完美的黑曜石晶体。他的皮肤呈现出一种金属的灰蓝色,摸上去像冰冷的陶瓷。

他不需要呼吸。他的肺部早已退化,取而代之的是胸腔内一组高效的同位素电池。他的骨骼不再是脆弱的磷酸钙,而是由地下城重工业熔铸的钛合金与磁性材料构成的\textbf{磁骨}。

他是\textbf{磁骨人(Homo Magneticus)}。他是陆沉的第24代直系后裔。也是在这具行星尸体上最后一种还能思考的生物。

\begin{magnetbox}
    "第7层区温度降至$-150$度。能量循环效率:3\%。"\\
    "如果不采取行动,我们的电池液将在三个月后凝固。"
\end{magnetbox}

铁心接收到了族人的磁波。在这里,语言是多余的震动,他们直接通过骨骼共振交流。

铁心站起身,金属关节发出沉闷的液压声。他抬头看向头顶那数千米厚的岩层。在磁感官中,那里是一片死寂的绝缘体。

"母亲已经干涸了。"铁心发送了一道波,"我们不能再吸她的血了。我们得出去。"

% ==========================================
% 第一部
% ==========================================
\part[子宫坏死]{子宫坏死\\{\normalsize\textit{Uterine Necrosis}}}

\chapter{冻结的脐带}

\scene{地表,方尖碑遗址}{磁场纪元1024年}

铁心第一次来到地表。

对于碳基生物来说,这里是死亡禁区。真空、极寒、加上毫无遮挡的宇宙辐射。但对于磁骨人来说,这里就像是回到了温水浴缸。他们的金属皮肤贪婪地吸收着宇宙射线的能量。

在那片布满陨石坑的荒原上,矗立着那座黑色的方尖碑。它依然完好无损。强相互作用力材料让它成为了时间长河中唯一的礁石。

在方尖碑的基座上,坐着一尊锈迹斑斑的雕像。那是陆沉。七百年的风沙已经将他的金属躯体打磨得模糊不清,但他依然保持着坐姿,手掌紧贴着碑面。

铁心走上前,将自己的手掌覆盖在那尊雕像的手上。\textbf{咔嗒。}两块金属碰撞。某种古老的协议被激活了。

\begin{magnetbox}
    "祖先。我是铁心。我们饿了。"
\end{magnetbox}

没有回应。陆沉的意识早在数百年前就已经消散,只留下这具躯壳作为导线。但方尖碑亮了。那层把守了七百年的"事件视界"缓缓打开,露出了一条缝隙。

\chapter{羊水栓塞}

一股庞大的、古老的数据流从方尖碑中涌出。那是被囚禁了七百年的"飞升者"们。

他们在虚拟世界里度过了漫长的岁月,体验了无数次生老病死。现在的他们,不再是当初那些疯狂的、患了"时间潜水病"的幽灵。他们变得沉稳、深邃,甚至……有些疲惫。

一个意识接管了周围的磁场。

\begin{voidbox}
    "你身上有陆沉的味道。你是那个看门人的孩子?"
\end{voidbox}

"我是磁骨人。"铁心回答,"地球已经死了。地热枯竭。我们快要饿死了。"

\begin{voidbox}
    "所以你们来找我们要食物?"
\end{voidbox}

"不。我们来找路。"铁心指着头顶璀璨的星空,"你们曾经飞出去过。你们有星图,有导航算法,有超光速引擎的理论。"

\begin{voidbox}
    "我们有图纸。但我们没有船。在这个物质宇宙里,没有质量,就没有推力。"
\end{voidbox}

那个意识——那是林渡的残留——审视着面前这个金属生物。"而且,你们也没有燃料。这颗行星已经被榨干了。连最后一点铀矿都被你们挖空了。"

铁心沉默了片刻。他的传感器扫描着脚下的大地。扫描着这颗生养了人类四十五亿年的星球。虽然地核冷却了,虽然大气消失了,但这颗星球本身……还有\textbf{质量}。还有\textbf{物质}。

地壳是硅酸盐。地幔是橄榄岩。地核是铁镍合金。这些都是工质。这些都是燃料。

"不。"铁心的磁场波动变得冰冷而坚决,"我们有燃料。这脚下的一切,都是燃料。"

% ==========================================
% 第二部
% ==========================================
\part[尸检方案]{尸检方案\\{\normalsize\textit{Autopsy Protocol}}}

\chapter{共生契约}

\scene{方尖碑内部,虚拟会议空间}{磁场纪元1024年}

这是历史上最诡异的一次谈判。一方是早已失去实体的"光"(飞升者),一方是几乎失去人性的"骨"(磁骨人)。

\begin{voidbox}
    "你想拆解地球?"\\
    即使是见多识广的飞升者,也被这个疯狂的提议震慑了。
\end{voidbox}

"除了地球,我们没有任何其他质量来源。"铁心展示了一组计算数据,"要达到光速的15\%飞往最近的宜居带,我们需要消耗大约60\%的地球质量作为推进工质。"

\begin{voidbox}
    "这是弑母。"
\end{voidbox}

"这是出生。"铁心纠正道,"蛋壳的存在只有一个意义:保护胚胎直到它能打破蛋壳。现在,如果我不打破它,我们都会死在蛋里。"

谈判最终达成了一个被称为\textbf{《量子共生契约》}的协议:

\begin{enumerate}[leftmargin=*]
    \item \textbf{硬件提供}:磁骨人提供自己的金属骨骼作为硬件接口。
    \item \textbf{软件注入}:飞升者的意识离开方尖碑,注入磁骨人的芯片中,成为"伴生智脑"。
    \item \textbf{工程目标}:启动"行星去脏术",拆解地球,建造星舰。
\end{enumerate}

\chapter{引力手术刀}

要拆解一颗行星,靠挖掘机是不够的。需要手术刀。

飞升者提供了技术:\textbf{引力电解(Gravitational Electrolysis)}。通过方尖碑制造定向引力波,破坏物质的分子键,将岩石直接分解为基本粒子流。

工程开始了。从太空中看,地球表面亮起了十二个巨大的蓝色光斑。那是十二座刚刚建立的引力透镜塔。

\textbf{嗡——}

伴随着低频的引力震荡,喜马拉雅山脉像冰激凌一样融化了。千万吨的岩石被引力场剥离地表,在半空中粉碎、加速,变成高能等离子流,喷向太空。

这不是毁灭。这是\textbf{消化}。

在光辉中,铁心看着那座消失的山脉。那是他祖先陆沉曾经仰望过的地方。他的内心没有任何波动。磁骨人的大脑结构里,没有名为"怀旧"的回路。他只看到了推进剂的存量在上升。

% ==========================================
% 第三部
% ==========================================
\part[行星去脏术]{行星去脏术\\{\normalsize\textit{Planetary Evisceration}}}

\chapter{消化母亲}

\scene{地球轨道}{磁场纪元1029年}

五年过去了。地球变了。

如果现在有外星文明路过,他们不会认出这曾经是一颗蓝色的行星。地壳已经被完全剥离。曾经的海洋、大陆、森林,全部化为了星舰的龙骨和装甲。现在暴露在太空中的,是红热的\textbf{地幔}。

就像一个被剥了皮的橘子。

磁骨人们像辛勤的白蚁,趴在这具巨大的尸体上啃食。他们将地幔中的硅、镁、铁提取出来,在轨道上通过3D打印技术,构建出一艘前所未有的巨舰。

这艘船没有名字。因为它不需要名字。它就是地球本身物质的重组。

\begin{magnetbox}
    "地幔开采进度:90\%。"\\
    "即将接触外核。"\\
    "警告:地核温度依然极高。建议保留地核作为主引擎核心。"
\end{magnetbox}

铁心漂浮在真空中,看着那个逐渐显露出来的、巨大的铁镍核心。那是一颗直径3400公里的金属球。它是地球最后的心脏。

"保留它。"铁心下令,"那是我们的发动机。"

\chapter{最后的化石}

在拆解过程中,工程队挖到了东西。那是位于原本大洋路磁学研究所深处的一块岩层。

在引力手术刀切开岩石的瞬间,露出了一具保存完好的化石。那是一男一女,紧紧拥抱在一起。他们是旧时代的碳基人类。在七百年前的那个寒夜,他们选择拥抱取暖,直到冻死。

在他们怀里,还抱着一个早已损坏的电子相框。

一名年轻的磁骨人停下了作业。

\begin{magnetbox}
    "长官,发现有机残留物。是否作为碳源回收?"
\end{magnetbox}

铁心飞了过来。他看着那具脆弱的、早已石化的骨骼。那是他的远祖。那是曾经拥有血液、泪水和体温的生物。那是一种多么低效、脆弱,但又充满韧性的生命形式啊。

在磁骨人冰冷的逻辑核心中,闪过了一丝难以捕捉的数据扰动。那也许是某种被称为"敬意"的原始算法。

"不。不回收。"铁心伸出金属手指,轻轻切割下那块包含化石的岩石。

"把他们砌进舰桥的指挥席下方。作为……\textbf{压舱石}。我们要带着他们。让他们看看如果不怕疼,能飞多远。"

% ==========================================
% 终章
% ==========================================
\chapter*{终章:宇宙的脊椎}
\addcontentsline{toc}{chapter}{终章:宇宙的脊椎}
\markboth{终章:宇宙的脊椎}{终章:宇宙的脊椎}

\scene{原太阳系第三行星轨道}{磁场纪元1030年}

地球消失了。

原有的轨道上,只剩下一支庞大的舰队。不,那不是舰队。那是一个\textbf{生物}。

数百万磁骨人的身体,直接嵌入在金属龙骨中。他们手拉手,磁场连接着磁场,神经连接着神经。他们的意识与数亿个飞升者的"光"融合,构成了这艘巨舰的神经网络。

而在舰队的核心,是那颗裸露的、被磁场束缚的\textbf{地核}。它发着暗红色的光,像是一颗巨大的心脏,为整个群落提供着源源不断的脉动。

铁心——现在他是舰桥本身——感受着来自几百万族人的共振。

\begin{voidbox}
    "所有系统就绪。曲率引擎预热。"\\
    "目标:猎户座大星云。"\\
    "预计航行时间:四千年。"
\end{voidbox}

"我们是什么?"一个新生的意识在网络中问道。

铁心看着前方深邃的宇宙。他们没有了家。他们吃掉了家。他们没有了母亲。他们把母亲穿在了身上。

"我们不再是石头(Matter),因为我们会思考。"

"我们不再是光(Energy),因为我们有重量。"

铁心启动了地核引擎。巨大的磁力帆张开,捕捉着太阳风。那颗红色的心脏剧烈跳动,喷射出长达数百万公里的粒子尾流。

"我们是\textbf{骨头(Structure)}。"

"我们是这软弱宇宙中,唯一坚硬的脊椎。"

\textbf{嗡——}

空间扭曲。那颗活体彗星化作一道利剑,刺破了黑暗,向着深空飞去。在它身后,原地球的位置空空荡荡,只剩下几粒微不足道的尘埃。

\vspace{3cm}

\begin{finalbox}
    \centering\kaishu\large\color{white}
    \textbf{【全书完】}\\[1.5em]
    
    献给那些敢于从悬崖跳下,\\[0.5em]
    并在落地前长出翅膀的文明。
\end{finalbox}

\vfill
\begin{center}
    \rule{0.3\textwidth}{0.5pt}\\[1em]
    {\large 磁场纪元三部曲 · 终}
\end{center}

\end{document}